
% Default to the notebook output style

    


% Inherit from the specified cell style.




    
\documentclass[11pt]{article}

    
    
    \usepackage[T1]{fontenc}
    % Nicer default font (+ math font) than Computer Modern for most use cases
    \usepackage{mathpazo}

    % Basic figure setup, for now with no caption control since it's done
    % automatically by Pandoc (which extracts ![](path) syntax from Markdown).
    \usepackage{graphicx}
    % We will generate all images so they have a width \maxwidth. This means
    % that they will get their normal width if they fit onto the page, but
    % are scaled down if they would overflow the margins.
    \makeatletter
    \def\maxwidth{\ifdim\Gin@nat@width>\linewidth\linewidth
    \else\Gin@nat@width\fi}
    \makeatother
    \let\Oldincludegraphics\includegraphics
    % Set max figure width to be 80% of text width, for now hardcoded.
    \renewcommand{\includegraphics}[1]{\Oldincludegraphics[width=.8\maxwidth]{#1}}
    % Ensure that by default, figures have no caption (until we provide a
    % proper Figure object with a Caption API and a way to capture that
    % in the conversion process - todo).
    \usepackage{caption}
    \DeclareCaptionLabelFormat{nolabel}{}
    \captionsetup{labelformat=nolabel}

    \usepackage{adjustbox} % Used to constrain images to a maximum size 
    \usepackage{xcolor} % Allow colors to be defined
    \usepackage{enumerate} % Needed for markdown enumerations to work
    \usepackage{geometry} % Used to adjust the document margins
    \usepackage{amsmath} % Equations
    \usepackage{amssymb} % Equations
    \usepackage{textcomp} % defines textquotesingle
    % Hack from http://tex.stackexchange.com/a/47451/13684:
    \AtBeginDocument{%
        \def\PYZsq{\textquotesingle}% Upright quotes in Pygmentized code
    }
    \usepackage{upquote} % Upright quotes for verbatim code
    \usepackage{eurosym} % defines \euro
    \usepackage[mathletters]{ucs} % Extended unicode (utf-8) support
    \usepackage[utf8x]{inputenc} % Allow utf-8 characters in the tex document
    \usepackage{fancyvrb} % verbatim replacement that allows latex
    \usepackage{grffile} % extends the file name processing of package graphics 
                         % to support a larger range 
    % The hyperref package gives us a pdf with properly built
    % internal navigation ('pdf bookmarks' for the table of contents,
    % internal cross-reference links, web links for URLs, etc.)
    \usepackage{hyperref}
    \usepackage{longtable} % longtable support required by pandoc >1.10
    \usepackage{booktabs}  % table support for pandoc > 1.12.2
    \usepackage[inline]{enumitem} % IRkernel/repr support (it uses the enumerate* environment)
    \usepackage[normalem]{ulem} % ulem is needed to support strikethroughs (\sout)
                                % normalem makes italics be italics, not underlines
    

    
    
    % Colors for the hyperref package
    \definecolor{urlcolor}{rgb}{0,.145,.698}
    \definecolor{linkcolor}{rgb}{.71,0.21,0.01}
    \definecolor{citecolor}{rgb}{.12,.54,.11}

    % ANSI colors
    \definecolor{ansi-black}{HTML}{3E424D}
    \definecolor{ansi-black-intense}{HTML}{282C36}
    \definecolor{ansi-red}{HTML}{E75C58}
    \definecolor{ansi-red-intense}{HTML}{B22B31}
    \definecolor{ansi-green}{HTML}{00A250}
    \definecolor{ansi-green-intense}{HTML}{007427}
    \definecolor{ansi-yellow}{HTML}{DDB62B}
    \definecolor{ansi-yellow-intense}{HTML}{B27D12}
    \definecolor{ansi-blue}{HTML}{208FFB}
    \definecolor{ansi-blue-intense}{HTML}{0065CA}
    \definecolor{ansi-magenta}{HTML}{D160C4}
    \definecolor{ansi-magenta-intense}{HTML}{A03196}
    \definecolor{ansi-cyan}{HTML}{60C6C8}
    \definecolor{ansi-cyan-intense}{HTML}{258F8F}
    \definecolor{ansi-white}{HTML}{C5C1B4}
    \definecolor{ansi-white-intense}{HTML}{A1A6B2}

    % commands and environments needed by pandoc snippets
    % extracted from the output of `pandoc -s`
    \providecommand{\tightlist}{%
      \setlength{\itemsep}{0pt}\setlength{\parskip}{0pt}}
    \DefineVerbatimEnvironment{Highlighting}{Verbatim}{commandchars=\\\{\}}
    % Add ',fontsize=\small' for more characters per line
    \newenvironment{Shaded}{}{}
    \newcommand{\KeywordTok}[1]{\textcolor[rgb]{0.00,0.44,0.13}{\textbf{{#1}}}}
    \newcommand{\DataTypeTok}[1]{\textcolor[rgb]{0.56,0.13,0.00}{{#1}}}
    \newcommand{\DecValTok}[1]{\textcolor[rgb]{0.25,0.63,0.44}{{#1}}}
    \newcommand{\BaseNTok}[1]{\textcolor[rgb]{0.25,0.63,0.44}{{#1}}}
    \newcommand{\FloatTok}[1]{\textcolor[rgb]{0.25,0.63,0.44}{{#1}}}
    \newcommand{\CharTok}[1]{\textcolor[rgb]{0.25,0.44,0.63}{{#1}}}
    \newcommand{\StringTok}[1]{\textcolor[rgb]{0.25,0.44,0.63}{{#1}}}
    \newcommand{\CommentTok}[1]{\textcolor[rgb]{0.38,0.63,0.69}{\textit{{#1}}}}
    \newcommand{\OtherTok}[1]{\textcolor[rgb]{0.00,0.44,0.13}{{#1}}}
    \newcommand{\AlertTok}[1]{\textcolor[rgb]{1.00,0.00,0.00}{\textbf{{#1}}}}
    \newcommand{\FunctionTok}[1]{\textcolor[rgb]{0.02,0.16,0.49}{{#1}}}
    \newcommand{\RegionMarkerTok}[1]{{#1}}
    \newcommand{\ErrorTok}[1]{\textcolor[rgb]{1.00,0.00,0.00}{\textbf{{#1}}}}
    \newcommand{\NormalTok}[1]{{#1}}
    
    % Additional commands for more recent versions of Pandoc
    \newcommand{\ConstantTok}[1]{\textcolor[rgb]{0.53,0.00,0.00}{{#1}}}
    \newcommand{\SpecialCharTok}[1]{\textcolor[rgb]{0.25,0.44,0.63}{{#1}}}
    \newcommand{\VerbatimStringTok}[1]{\textcolor[rgb]{0.25,0.44,0.63}{{#1}}}
    \newcommand{\SpecialStringTok}[1]{\textcolor[rgb]{0.73,0.40,0.53}{{#1}}}
    \newcommand{\ImportTok}[1]{{#1}}
    \newcommand{\DocumentationTok}[1]{\textcolor[rgb]{0.73,0.13,0.13}{\textit{{#1}}}}
    \newcommand{\AnnotationTok}[1]{\textcolor[rgb]{0.38,0.63,0.69}{\textbf{\textit{{#1}}}}}
    \newcommand{\CommentVarTok}[1]{\textcolor[rgb]{0.38,0.63,0.69}{\textbf{\textit{{#1}}}}}
    \newcommand{\VariableTok}[1]{\textcolor[rgb]{0.10,0.09,0.49}{{#1}}}
    \newcommand{\ControlFlowTok}[1]{\textcolor[rgb]{0.00,0.44,0.13}{\textbf{{#1}}}}
    \newcommand{\OperatorTok}[1]{\textcolor[rgb]{0.40,0.40,0.40}{{#1}}}
    \newcommand{\BuiltInTok}[1]{{#1}}
    \newcommand{\ExtensionTok}[1]{{#1}}
    \newcommand{\PreprocessorTok}[1]{\textcolor[rgb]{0.74,0.48,0.00}{{#1}}}
    \newcommand{\AttributeTok}[1]{\textcolor[rgb]{0.49,0.56,0.16}{{#1}}}
    \newcommand{\InformationTok}[1]{\textcolor[rgb]{0.38,0.63,0.69}{\textbf{\textit{{#1}}}}}
    \newcommand{\WarningTok}[1]{\textcolor[rgb]{0.38,0.63,0.69}{\textbf{\textit{{#1}}}}}
    
    
    % Define a nice break command that doesn't care if a line doesn't already
    % exist.
    \def\br{\hspace*{\fill} \\* }
    % Math Jax compatability definitions
    \def\gt{>}
    \def\lt{<}
    % Document parameters
    \title{Earth Launcher Essay}
    
    
    

    % Pygments definitions
    
\makeatletter
\def\PY@reset{\let\PY@it=\relax \let\PY@bf=\relax%
    \let\PY@ul=\relax \let\PY@tc=\relax%
    \let\PY@bc=\relax \let\PY@ff=\relax}
\def\PY@tok#1{\csname PY@tok@#1\endcsname}
\def\PY@toks#1+{\ifx\relax#1\empty\else%
    \PY@tok{#1}\expandafter\PY@toks\fi}
\def\PY@do#1{\PY@bc{\PY@tc{\PY@ul{%
    \PY@it{\PY@bf{\PY@ff{#1}}}}}}}
\def\PY#1#2{\PY@reset\PY@toks#1+\relax+\PY@do{#2}}

\expandafter\def\csname PY@tok@w\endcsname{\def\PY@tc##1{\textcolor[rgb]{0.73,0.73,0.73}{##1}}}
\expandafter\def\csname PY@tok@c\endcsname{\let\PY@it=\textit\def\PY@tc##1{\textcolor[rgb]{0.25,0.50,0.50}{##1}}}
\expandafter\def\csname PY@tok@cp\endcsname{\def\PY@tc##1{\textcolor[rgb]{0.74,0.48,0.00}{##1}}}
\expandafter\def\csname PY@tok@k\endcsname{\let\PY@bf=\textbf\def\PY@tc##1{\textcolor[rgb]{0.00,0.50,0.00}{##1}}}
\expandafter\def\csname PY@tok@kp\endcsname{\def\PY@tc##1{\textcolor[rgb]{0.00,0.50,0.00}{##1}}}
\expandafter\def\csname PY@tok@kt\endcsname{\def\PY@tc##1{\textcolor[rgb]{0.69,0.00,0.25}{##1}}}
\expandafter\def\csname PY@tok@o\endcsname{\def\PY@tc##1{\textcolor[rgb]{0.40,0.40,0.40}{##1}}}
\expandafter\def\csname PY@tok@ow\endcsname{\let\PY@bf=\textbf\def\PY@tc##1{\textcolor[rgb]{0.67,0.13,1.00}{##1}}}
\expandafter\def\csname PY@tok@nb\endcsname{\def\PY@tc##1{\textcolor[rgb]{0.00,0.50,0.00}{##1}}}
\expandafter\def\csname PY@tok@nf\endcsname{\def\PY@tc##1{\textcolor[rgb]{0.00,0.00,1.00}{##1}}}
\expandafter\def\csname PY@tok@nc\endcsname{\let\PY@bf=\textbf\def\PY@tc##1{\textcolor[rgb]{0.00,0.00,1.00}{##1}}}
\expandafter\def\csname PY@tok@nn\endcsname{\let\PY@bf=\textbf\def\PY@tc##1{\textcolor[rgb]{0.00,0.00,1.00}{##1}}}
\expandafter\def\csname PY@tok@ne\endcsname{\let\PY@bf=\textbf\def\PY@tc##1{\textcolor[rgb]{0.82,0.25,0.23}{##1}}}
\expandafter\def\csname PY@tok@nv\endcsname{\def\PY@tc##1{\textcolor[rgb]{0.10,0.09,0.49}{##1}}}
\expandafter\def\csname PY@tok@no\endcsname{\def\PY@tc##1{\textcolor[rgb]{0.53,0.00,0.00}{##1}}}
\expandafter\def\csname PY@tok@nl\endcsname{\def\PY@tc##1{\textcolor[rgb]{0.63,0.63,0.00}{##1}}}
\expandafter\def\csname PY@tok@ni\endcsname{\let\PY@bf=\textbf\def\PY@tc##1{\textcolor[rgb]{0.60,0.60,0.60}{##1}}}
\expandafter\def\csname PY@tok@na\endcsname{\def\PY@tc##1{\textcolor[rgb]{0.49,0.56,0.16}{##1}}}
\expandafter\def\csname PY@tok@nt\endcsname{\let\PY@bf=\textbf\def\PY@tc##1{\textcolor[rgb]{0.00,0.50,0.00}{##1}}}
\expandafter\def\csname PY@tok@nd\endcsname{\def\PY@tc##1{\textcolor[rgb]{0.67,0.13,1.00}{##1}}}
\expandafter\def\csname PY@tok@s\endcsname{\def\PY@tc##1{\textcolor[rgb]{0.73,0.13,0.13}{##1}}}
\expandafter\def\csname PY@tok@sd\endcsname{\let\PY@it=\textit\def\PY@tc##1{\textcolor[rgb]{0.73,0.13,0.13}{##1}}}
\expandafter\def\csname PY@tok@si\endcsname{\let\PY@bf=\textbf\def\PY@tc##1{\textcolor[rgb]{0.73,0.40,0.53}{##1}}}
\expandafter\def\csname PY@tok@se\endcsname{\let\PY@bf=\textbf\def\PY@tc##1{\textcolor[rgb]{0.73,0.40,0.13}{##1}}}
\expandafter\def\csname PY@tok@sr\endcsname{\def\PY@tc##1{\textcolor[rgb]{0.73,0.40,0.53}{##1}}}
\expandafter\def\csname PY@tok@ss\endcsname{\def\PY@tc##1{\textcolor[rgb]{0.10,0.09,0.49}{##1}}}
\expandafter\def\csname PY@tok@sx\endcsname{\def\PY@tc##1{\textcolor[rgb]{0.00,0.50,0.00}{##1}}}
\expandafter\def\csname PY@tok@m\endcsname{\def\PY@tc##1{\textcolor[rgb]{0.40,0.40,0.40}{##1}}}
\expandafter\def\csname PY@tok@gh\endcsname{\let\PY@bf=\textbf\def\PY@tc##1{\textcolor[rgb]{0.00,0.00,0.50}{##1}}}
\expandafter\def\csname PY@tok@gu\endcsname{\let\PY@bf=\textbf\def\PY@tc##1{\textcolor[rgb]{0.50,0.00,0.50}{##1}}}
\expandafter\def\csname PY@tok@gd\endcsname{\def\PY@tc##1{\textcolor[rgb]{0.63,0.00,0.00}{##1}}}
\expandafter\def\csname PY@tok@gi\endcsname{\def\PY@tc##1{\textcolor[rgb]{0.00,0.63,0.00}{##1}}}
\expandafter\def\csname PY@tok@gr\endcsname{\def\PY@tc##1{\textcolor[rgb]{1.00,0.00,0.00}{##1}}}
\expandafter\def\csname PY@tok@ge\endcsname{\let\PY@it=\textit}
\expandafter\def\csname PY@tok@gs\endcsname{\let\PY@bf=\textbf}
\expandafter\def\csname PY@tok@gp\endcsname{\let\PY@bf=\textbf\def\PY@tc##1{\textcolor[rgb]{0.00,0.00,0.50}{##1}}}
\expandafter\def\csname PY@tok@go\endcsname{\def\PY@tc##1{\textcolor[rgb]{0.53,0.53,0.53}{##1}}}
\expandafter\def\csname PY@tok@gt\endcsname{\def\PY@tc##1{\textcolor[rgb]{0.00,0.27,0.87}{##1}}}
\expandafter\def\csname PY@tok@err\endcsname{\def\PY@bc##1{\setlength{\fboxsep}{0pt}\fcolorbox[rgb]{1.00,0.00,0.00}{1,1,1}{\strut ##1}}}
\expandafter\def\csname PY@tok@kc\endcsname{\let\PY@bf=\textbf\def\PY@tc##1{\textcolor[rgb]{0.00,0.50,0.00}{##1}}}
\expandafter\def\csname PY@tok@kd\endcsname{\let\PY@bf=\textbf\def\PY@tc##1{\textcolor[rgb]{0.00,0.50,0.00}{##1}}}
\expandafter\def\csname PY@tok@kn\endcsname{\let\PY@bf=\textbf\def\PY@tc##1{\textcolor[rgb]{0.00,0.50,0.00}{##1}}}
\expandafter\def\csname PY@tok@kr\endcsname{\let\PY@bf=\textbf\def\PY@tc##1{\textcolor[rgb]{0.00,0.50,0.00}{##1}}}
\expandafter\def\csname PY@tok@bp\endcsname{\def\PY@tc##1{\textcolor[rgb]{0.00,0.50,0.00}{##1}}}
\expandafter\def\csname PY@tok@fm\endcsname{\def\PY@tc##1{\textcolor[rgb]{0.00,0.00,1.00}{##1}}}
\expandafter\def\csname PY@tok@vc\endcsname{\def\PY@tc##1{\textcolor[rgb]{0.10,0.09,0.49}{##1}}}
\expandafter\def\csname PY@tok@vg\endcsname{\def\PY@tc##1{\textcolor[rgb]{0.10,0.09,0.49}{##1}}}
\expandafter\def\csname PY@tok@vi\endcsname{\def\PY@tc##1{\textcolor[rgb]{0.10,0.09,0.49}{##1}}}
\expandafter\def\csname PY@tok@vm\endcsname{\def\PY@tc##1{\textcolor[rgb]{0.10,0.09,0.49}{##1}}}
\expandafter\def\csname PY@tok@sa\endcsname{\def\PY@tc##1{\textcolor[rgb]{0.73,0.13,0.13}{##1}}}
\expandafter\def\csname PY@tok@sb\endcsname{\def\PY@tc##1{\textcolor[rgb]{0.73,0.13,0.13}{##1}}}
\expandafter\def\csname PY@tok@sc\endcsname{\def\PY@tc##1{\textcolor[rgb]{0.73,0.13,0.13}{##1}}}
\expandafter\def\csname PY@tok@dl\endcsname{\def\PY@tc##1{\textcolor[rgb]{0.73,0.13,0.13}{##1}}}
\expandafter\def\csname PY@tok@s2\endcsname{\def\PY@tc##1{\textcolor[rgb]{0.73,0.13,0.13}{##1}}}
\expandafter\def\csname PY@tok@sh\endcsname{\def\PY@tc##1{\textcolor[rgb]{0.73,0.13,0.13}{##1}}}
\expandafter\def\csname PY@tok@s1\endcsname{\def\PY@tc##1{\textcolor[rgb]{0.73,0.13,0.13}{##1}}}
\expandafter\def\csname PY@tok@mb\endcsname{\def\PY@tc##1{\textcolor[rgb]{0.40,0.40,0.40}{##1}}}
\expandafter\def\csname PY@tok@mf\endcsname{\def\PY@tc##1{\textcolor[rgb]{0.40,0.40,0.40}{##1}}}
\expandafter\def\csname PY@tok@mh\endcsname{\def\PY@tc##1{\textcolor[rgb]{0.40,0.40,0.40}{##1}}}
\expandafter\def\csname PY@tok@mi\endcsname{\def\PY@tc##1{\textcolor[rgb]{0.40,0.40,0.40}{##1}}}
\expandafter\def\csname PY@tok@il\endcsname{\def\PY@tc##1{\textcolor[rgb]{0.40,0.40,0.40}{##1}}}
\expandafter\def\csname PY@tok@mo\endcsname{\def\PY@tc##1{\textcolor[rgb]{0.40,0.40,0.40}{##1}}}
\expandafter\def\csname PY@tok@ch\endcsname{\let\PY@it=\textit\def\PY@tc##1{\textcolor[rgb]{0.25,0.50,0.50}{##1}}}
\expandafter\def\csname PY@tok@cm\endcsname{\let\PY@it=\textit\def\PY@tc##1{\textcolor[rgb]{0.25,0.50,0.50}{##1}}}
\expandafter\def\csname PY@tok@cpf\endcsname{\let\PY@it=\textit\def\PY@tc##1{\textcolor[rgb]{0.25,0.50,0.50}{##1}}}
\expandafter\def\csname PY@tok@c1\endcsname{\let\PY@it=\textit\def\PY@tc##1{\textcolor[rgb]{0.25,0.50,0.50}{##1}}}
\expandafter\def\csname PY@tok@cs\endcsname{\let\PY@it=\textit\def\PY@tc##1{\textcolor[rgb]{0.25,0.50,0.50}{##1}}}

\def\PYZbs{\char`\\}
\def\PYZus{\char`\_}
\def\PYZob{\char`\{}
\def\PYZcb{\char`\}}
\def\PYZca{\char`\^}
\def\PYZam{\char`\&}
\def\PYZlt{\char`\<}
\def\PYZgt{\char`\>}
\def\PYZsh{\char`\#}
\def\PYZpc{\char`\%}
\def\PYZdl{\char`\$}
\def\PYZhy{\char`\-}
\def\PYZsq{\char`\'}
\def\PYZdq{\char`\"}
\def\PYZti{\char`\~}
% for compatibility with earlier versions
\def\PYZat{@}
\def\PYZlb{[}
\def\PYZrb{]}
\makeatother


    % Exact colors from NB
    \definecolor{incolor}{rgb}{0.0, 0.0, 0.5}
    \definecolor{outcolor}{rgb}{0.545, 0.0, 0.0}



    
    % Prevent overflowing lines due to hard-to-break entities
    \sloppy 
    % Setup hyperref package
    \hypersetup{
      breaklinks=true,  % so long urls are correctly broken across lines
      colorlinks=true,
      urlcolor=urlcolor,
      linkcolor=linkcolor,
      citecolor=citecolor,
      }
    % Slightly bigger margins than the latex defaults
    
    \geometry{verbose,tmargin=1in,bmargin=1in,lmargin=1in,rmargin=1in}
    
    

    \begin{document}
    
    
    \maketitle
    
    

    
    \hypertarget{earth-launcher}{%
\section{Earth Launcher}\label{earth-launcher}}

    The following code configures the notebook and imports relevant
libraries

    \begin{Verbatim}[commandchars=\\\{\}]
{\color{incolor}In [{\color{incolor}1}]:} \PY{c+c1}{\PYZsh{} Configure Jupyter so figures appear in the notebook}
        \PY{o}{\PYZpc{}}\PY{k}{matplotlib} inline
        
        \PY{c+c1}{\PYZsh{} Configure Jupyter to display the assigned value after an assignment}
        \PY{o}{\PYZpc{}}\PY{k}{config} InteractiveShell.ast\PYZus{}node\PYZus{}interactivity=\PYZsq{}last\PYZus{}expr\PYZus{}or\PYZus{}assign\PYZsq{}
        \PY{c+c1}{\PYZsh{} import functions from the modsim.py module}
        \PY{k+kn}{from} \PY{n+nn}{modsim} \PY{k}{import} \PY{o}{*}
        \PY{k+kn}{import} \PY{n+nn}{numpy} \PY{k}{as} \PY{n+nn}{np}
\end{Verbatim}


    \hypertarget{question}{%
\section{Question}\label{question}}

    What are the design requirements for a railgun such that firing the
railgun can accelerate the earth to the escape velocity of the sun?

    The above modeling question might seem highly rediculous, hovever
similar situations are being studied as spacecraft launch systems. One
example is the electromagnetic mass driver which is a hypothetical
concept of using a electromagnetically powered cannon, like a railgun or
coilgun, to accelerate spacecraft for launch (see Applications section
of the following page: https://en.wikipedia.org/wiki/Railgun or the
dedicated page on the subject
https://en.wikipedia.org/wiki/Mass\_driver) This project models the
reverse situation where the momentum gennerated by the shot is intended
to accelerate the earth rather than to accelerate a projectile.

    \begin{Verbatim}[commandchars=\\\{\}]
{\color{incolor}In [{\color{incolor}2}]:} \PY{n}{m} \PY{o}{=} \PY{n}{UNITS}\PY{o}{.}\PY{n}{meter}
        \PY{n}{s} \PY{o}{=} \PY{n}{UNITS}\PY{o}{.}\PY{n}{second}
        \PY{n}{kg} \PY{o}{=} \PY{n}{UNITS}\PY{o}{.}\PY{n}{kilogram}
        \PY{n}{A} \PY{o}{=} \PY{n}{UNITS}\PY{o}{.}\PY{n}{ampere}
        \PY{n}{V} \PY{o}{=} \PY{n}{UNITS}\PY{o}{.}\PY{n}{volts}
        \PY{n}{Ohm} \PY{o}{=} \PY{n}{UNITS}\PY{o}{.}\PY{n}{ohm}
        \PY{n}{N} \PY{o}{=} \PY{n}{UNITS}\PY{o}{.}\PY{n}{newton}
        \PY{n}{U\PYZus{}mu} \PY{o}{=} \PY{n}{N} \PY{o}{/} \PY{n}{A}\PY{o}{*}\PY{o}{*}\PY{l+m+mi}{2}
        \PY{k+kc}{None}
\end{Verbatim}


    \begin{Verbatim}[commandchars=\\\{\}]
{\color{incolor}In [{\color{incolor}3}]:} \PY{n}{params\PYZus{}railgun} \PY{o}{=} \PY{n}{Params}\PY{p}{(}\PY{n}{x} \PY{o}{=} \PY{l+m+mi}{0} \PY{o}{*} \PY{n}{m}\PY{p}{,} 
                                \PY{n}{d} \PY{o}{=} \PY{l+m+mi}{1} \PY{o}{*} \PY{n}{m}\PY{p}{,}
                                \PY{n}{r} \PY{o}{=} \PY{l+m+mf}{0.1} \PY{o}{*} \PY{n}{m}\PY{p}{,}
                                \PY{n}{L} \PY{o}{=} \PY{l+m+mi}{100} \PY{o}{*} \PY{n}{m}\PY{p}{,}
                                \PY{n}{mass} \PY{o}{=} \PY{l+m+mi}{1} \PY{o}{*} \PY{n}{kg}\PY{p}{,}
                                \PY{n}{Vs} \PY{o}{=} \PY{l+m+mi}{1000} \PY{o}{*} \PY{n}{V}\PY{p}{,}
                                \PY{n}{R} \PY{o}{=} \PY{l+m+mf}{0.01} \PY{o}{*} \PY{n}{Ohm}\PY{p}{,}
                                \PY{n}{mu\PYZus{}0} \PY{o}{=} \PY{l+m+mf}{1.257e\PYZhy{}6} \PY{o}{*} \PY{n}{U\PYZus{}mu}\PY{p}{,}
                                \PY{n}{t\PYZus{}end} \PY{o}{=} \PY{l+m+mi}{100} \PY{o}{*} \PY{n}{s}\PY{p}{)}
\end{Verbatim}


\begin{Verbatim}[commandchars=\\\{\}]
{\color{outcolor}Out[{\color{outcolor}3}]:} x                               0 meter
        d                               1 meter
        r                             0.1 meter
        L                             100 meter
        mass                         1 kilogram
        Vs                            1000 volt
        R                              0.01 ohm
        mu\_0     1.257e-06 newton / ampere ** 2
        t\_end                        100 second
        dtype: object
\end{Verbatim}
            
    \begin{Verbatim}[commandchars=\\\{\}]
{\color{incolor}In [{\color{incolor}4}]:} \PY{k}{def} \PY{n+nf}{make\PYZus{}railgun\PYZus{}system}\PY{p}{(}\PY{n}{params}\PY{p}{)}\PY{p}{:}
            \PY{l+s+sd}{\PYZdq{}\PYZdq{}\PYZdq{}}
        \PY{l+s+sd}{    Make a system object.}
        \PY{l+s+sd}{    }
        \PY{l+s+sd}{    Parameters:}
        \PY{l+s+sd}{        params: Params object containing following:}
        \PY{l+s+sd}{            x: Initial position of armature (in meters)}
        \PY{l+s+sd}{            d: Separation distance between two rails (in meters)}
        \PY{l+s+sd}{            r: Radii of the rails (in meters)}
        \PY{l+s+sd}{            L: Length armature has to travel along rails before launch (in meters)}
        \PY{l+s+sd}{            mass: Mass of armature (in kilograms)}
        \PY{l+s+sd}{            Vs: (Initial) Voltage supply magnitude (in volts)}
        \PY{l+s+sd}{            R: Resistance of railgun circuit (in ohms)}
        \PY{l+s+sd}{            mu\PYZus{}0: Vacuum permeability constant (in N/A\PYZca{}2)}
        \PY{l+s+sd}{            t\PYZus{}end: Simulation end time (in seconds)}
        \PY{l+s+sd}{               }
        \PY{l+s+sd}{    returns: System object containing}
        \PY{l+s+sd}{        params values}
        \PY{l+s+sd}{        c\PYZus{}mag: Calculated magnetic force coefficient (in N/A\PYZca{}2)}
        \PY{l+s+sd}{        init: Initial state object containing:}
        \PY{l+s+sd}{            x: Position of armature (initialized to params.x) (in meters)}
        \PY{l+s+sd}{            v: Velocity of armature (initialized to 0) (in meters)}
        \PY{l+s+sd}{    \PYZdq{}\PYZdq{}\PYZdq{}}
            
            \PY{c+c1}{\PYZsh{} make the initial state}
            \PY{n}{init} \PY{o}{=} \PY{n}{State}\PY{p}{(}\PY{n}{x}\PY{o}{=}\PY{n}{params}\PY{o}{.}\PY{n}{x}\PY{p}{,} \PY{n}{v}\PY{o}{=}\PY{l+m+mi}{0}\PY{p}{)}
            
            \PY{c+c1}{\PYZsh{} Determine magnetic force coefficient}
            \PY{n}{c\PYZus{}mag} \PY{o}{=} \PY{n}{params}\PY{o}{.}\PY{n}{mu\PYZus{}0} \PY{o}{*} \PY{n}{log}\PY{p}{(}\PY{p}{(}\PY{n}{params}\PY{o}{.}\PY{n}{d} \PY{o}{\PYZhy{}} \PY{n}{params}\PY{o}{.}\PY{n}{r}\PY{p}{)} \PY{o}{/} \PY{n}{params}\PY{o}{.}\PY{n}{r}\PY{p}{)} \PY{o}{/} \PY{p}{(}\PY{l+m+mi}{2} \PY{o}{*} \PY{n}{pi}\PY{p}{)}
            
            \PY{k}{return} \PY{n}{System}\PY{p}{(}\PY{n}{params}\PY{p}{,} \PY{n}{c\PYZus{}mag} \PY{o}{=} \PY{n}{c\PYZus{}mag}\PY{p}{,} \PY{n}{init}\PY{o}{=}\PY{n}{init}\PY{p}{)}
\end{Verbatim}


    \begin{Verbatim}[commandchars=\\\{\}]
{\color{incolor}In [{\color{incolor}5}]:} \PY{k}{def} \PY{n+nf}{mag\PYZus{}force}\PY{p}{(}\PY{n}{I}\PY{p}{,} \PY{n}{system}\PY{p}{)}\PY{p}{:}
            \PY{l+s+sd}{\PYZdq{}\PYZdq{}\PYZdq{}}
        \PY{l+s+sd}{    Determines magnetic force on armature due to magnetic loop of railgun. Assumes 1D system.}
        \PY{l+s+sd}{    }
        \PY{l+s+sd}{    Parameters:}
        \PY{l+s+sd}{        I: Current through the circuit at time of function call. (in amperes)}
        \PY{l+s+sd}{        system: System object containing following system parameters (relevant listed):}
        \PY{l+s+sd}{            c\PYZus{}mag: Calculated magnetic force coefficient (in N/A\PYZca{}2)}
        \PY{l+s+sd}{               }
        \PY{l+s+sd}{    returns: Magnitude of magnetic force}
        \PY{l+s+sd}{    \PYZdq{}\PYZdq{}\PYZdq{}}
            \PY{k}{return} \PY{n}{system}\PY{o}{.}\PY{n}{c\PYZus{}mag} \PY{o}{*} \PY{n}{I}\PY{o}{*}\PY{o}{*}\PY{l+m+mi}{2}
\end{Verbatim}


    \begin{Verbatim}[commandchars=\\\{\}]
{\color{incolor}In [{\color{incolor}6}]:} \PY{k}{def} \PY{n+nf}{railgun\PYZus{}slope\PYZus{}func}\PY{p}{(}\PY{n}{state}\PY{p}{,} \PY{n}{t}\PY{p}{,} \PY{n}{system}\PY{p}{)}\PY{p}{:}
            \PY{l+s+sd}{\PYZdq{}\PYZdq{}\PYZdq{}}
        \PY{l+s+sd}{    Computes derivatives of the state variables. }
        \PY{l+s+sd}{    Invokes function mag\PYZus{}force to determine magnetic force applied on armature.}
        \PY{l+s+sd}{    }
        \PY{l+s+sd}{    Parameters:}
        \PY{l+s+sd}{        state: Current State object at time of function call.}
        \PY{l+s+sd}{            x: Position of armature (initialized to params.x) (in meters)}
        \PY{l+s+sd}{            v: Velocity of armature (initialized to 0) (in meters)}
        \PY{l+s+sd}{        system: System object containing following system parameters (relevant listed):}
        \PY{l+s+sd}{            Vs: Magnitude of voltage supply to circuit (in volts)}
        \PY{l+s+sd}{            R: Resistance of the circuit (in ohms)}
        \PY{l+s+sd}{               }
        \PY{l+s+sd}{    returns: Magnitude of magnetic force (in newtons)}
        \PY{l+s+sd}{    \PYZdq{}\PYZdq{}\PYZdq{}}
            \PY{n}{x}\PY{p}{,} \PY{n}{v} \PY{o}{=} \PY{n}{state}
        
            \PY{n}{I} \PY{o}{=} \PY{n}{system}\PY{o}{.}\PY{n}{Vs} \PY{o}{/} \PY{n}{system}\PY{o}{.}\PY{n}{R}
            
            \PY{n}{F\PYZus{}mag} \PY{o}{=} \PY{n}{mag\PYZus{}force}\PY{p}{(}\PY{n}{I}\PY{p}{,} \PY{n}{system}\PY{p}{)}
            \PY{n}{a} \PY{o}{=} \PY{n}{F\PYZus{}mag} \PY{o}{/} \PY{n}{system}\PY{o}{.}\PY{n}{mass}
            
            \PY{k}{return} \PY{n}{v}\PY{p}{,} \PY{n}{a}
\end{Verbatim}


    \begin{Verbatim}[commandchars=\\\{\}]
{\color{incolor}In [{\color{incolor}7}]:} \PY{k}{def} \PY{n+nf}{event\PYZus{}func\PYZus{}railgun}\PY{p}{(}\PY{n}{state}\PY{p}{,} \PY{n}{t}\PY{p}{,} \PY{n}{system}\PY{p}{)}\PY{p}{:}
            \PY{l+s+sd}{\PYZdq{}\PYZdq{}\PYZdq{}}
        \PY{l+s+sd}{    Stop when the position is equal to the length of rail that can be traveled.}
        \PY{l+s+sd}{        (stop when the distance left to travel is equal to 0)}
        \PY{l+s+sd}{    }
        \PY{l+s+sd}{    Parameters:}
        \PY{l+s+sd}{        state: Current State object at time of function call.}
        \PY{l+s+sd}{            x: Position of armature (initialized to params.x) (in meters)}
        \PY{l+s+sd}{            v: Velocity of armature (initialized to 0) (in meters)}
        \PY{l+s+sd}{        system: System object containing following system parameters (relevant listed):}
        \PY{l+s+sd}{            L: Length armature has to travel along rails before launch (in meters)}
        \PY{l+s+sd}{               }
        \PY{l+s+sd}{    returns: Distance the armature has left to travel.}
        \PY{l+s+sd}{    \PYZdq{}\PYZdq{}\PYZdq{}}
            \PY{n}{x}\PY{p}{,} \PY{n}{v} \PY{o}{=} \PY{n}{state}
            \PY{k}{return} \PY{n}{x} \PY{o}{\PYZhy{}} \PY{n}{system}\PY{o}{.}\PY{n}{L}
\end{Verbatim}


    \begin{Verbatim}[commandchars=\\\{\}]
{\color{incolor}In [{\color{incolor}8}]:} \PY{k}{def} \PY{n+nf}{sim\PYZus{}railgun}\PY{p}{(}\PY{n}{mass}\PY{p}{,} \PY{n}{v\PYZus{}supply}\PY{p}{,} \PY{n}{params}\PY{p}{)}\PY{p}{:}  
            \PY{l+s+sd}{\PYZdq{}\PYZdq{}\PYZdq{}}
        \PY{l+s+sd}{    Simulate the railgun model.}
        \PY{l+s+sd}{    }
        \PY{l+s+sd}{    Parameters:}
        \PY{l+s+sd}{        mass: Mass of armature (in kilograms)}
        \PY{l+s+sd}{        params: Params object containing following:}
        \PY{l+s+sd}{            x: Initial position of armature (in meters)}
        \PY{l+s+sd}{            d: Separation distance between two rails (in meters)}
        \PY{l+s+sd}{            r: Radii of the rails (in meters)}
        \PY{l+s+sd}{            L: Length armature has to travel along rails before launch (in meters)}
        \PY{l+s+sd}{            mass: Mass of armature (in kilograms)}
        \PY{l+s+sd}{            Vs: (Initial) Voltage supply magnitude (in volts)}
        \PY{l+s+sd}{            R: Resistance of railgun circuit (in ohms)}
        \PY{l+s+sd}{            mu\PYZus{}0: Vacuum permeability constant (in N/A\PYZca{}2)}
        \PY{l+s+sd}{            t\PYZus{}end: Simulation end time (in seconds)}
        \PY{l+s+sd}{               }
        \PY{l+s+sd}{    returns:}
        \PY{l+s+sd}{        x: Final position of the armature.}
        \PY{l+s+sd}{        v: Final velocity of the armature.}
        \PY{l+s+sd}{    \PYZdq{}\PYZdq{}\PYZdq{}}
            \PY{n}{params} \PY{o}{=} \PY{n}{Params}\PY{p}{(}\PY{n}{params}\PY{p}{,} \PY{n}{mass}\PY{o}{=}\PY{n}{mass}\PY{p}{,} \PY{n}{Vs} \PY{o}{=} \PY{n}{v\PYZus{}supply}\PY{p}{)}
            \PY{n}{system} \PY{o}{=} \PY{n}{make\PYZus{}railgun\PYZus{}system}\PY{p}{(}\PY{n}{params}\PY{p}{)}
            \PY{n}{results}\PY{p}{,} \PY{n}{details} \PY{o}{=} \PY{n}{run\PYZus{}ode\PYZus{}solver}\PY{p}{(}\PY{n}{system}\PY{p}{,} \PY{n}{railgun\PYZus{}slope\PYZus{}func}\PY{p}{,} \PY{n}{events}\PY{o}{=}\PY{n}{event\PYZus{}func\PYZus{}railgun}\PY{p}{)}
            \PY{k}{return} \PY{n}{results}\PY{o}{.}\PY{n}{x}\PY{p}{,} \PY{n}{results}\PY{o}{.}\PY{n}{v}
\end{Verbatim}


    The following function runs a simulation of firing the railgun and the
effect which this action has on the orbit of the earth. The function
returns a time frame describing the earth's position and velocity at
various points in time after firing the railgun.

    \begin{Verbatim}[commandchars=\\\{\}]
{\color{incolor}In [{\color{incolor}9}]:} \PY{k}{def} \PY{n+nf}{launch\PYZus{}simulation}\PY{p}{(}\PY{n}{mass\PYZus{}armature}\PY{p}{,} \PY{n}{railgun\PYZus{}voltage}\PY{p}{,} \PY{n}{railgun\PYZus{}params}\PY{p}{,} 
                              \PY{n}{sim\PYZus{}t\PYZus{}end}\PY{p}{,} \PY{n}{orbital\PYZus{}results}\PY{p}{)}\PY{p}{:}
            \PY{c+c1}{\PYZsh{} Simulate railgun launch}
            \PY{n}{rail\PYZus{}x}\PY{p}{,} \PY{n}{rail\PYZus{}v} \PY{o}{=} \PY{n}{sim\PYZus{}railgun}\PY{p}{(}\PY{n}{mass\PYZus{}armature}\PY{p}{,} \PY{n}{railgun\PYZus{}voltage}\PY{p}{,} \PY{n}{railgun\PYZus{}params}\PY{p}{)}
            
            \PY{n}{final\PYZus{}armature\PYZus{}speed} \PY{o}{=} \PY{n}{get\PYZus{}last\PYZus{}value}\PY{p}{(}\PY{n}{rail\PYZus{}v}\PY{p}{)}
            \PY{c+c1}{\PYZsh{} Simulate after\PYZhy{}effects of railgun launch}
            \PY{n}{shot\PYZus{}results} \PY{o}{=} \PY{n}{run\PYZus{}simulation\PYZus{}after\PYZus{}cannon}\PY{p}{(}\PY{n}{mass\PYZus{}armature}\PY{p}{,} \PY{n}{final\PYZus{}armature\PYZus{}speed}\PY{p}{,} 
                                                       \PY{n}{sim\PYZus{}t\PYZus{}end}\PY{p}{,} \PY{n}{orbital\PYZus{}results}\PY{p}{)}
            
            \PY{k}{return} \PY{n}{shot\PYZus{}results}
\end{Verbatim}


    \hypertarget{planetary-motion-model-code}{%
\section{Planetary Motion Model
code}\label{planetary-motion-model-code}}

The following functions describe the planitary motion simulation

    The following code describes the main function which is used in running
simulations of a planet in orbit around the sun. This function returns a
time frame detailing the velocity and position of the plenet at varous
points in time.

    \begin{Verbatim}[commandchars=\\\{\}]
{\color{incolor}In [{\color{incolor}10}]:} \PY{k}{def} \PY{n+nf}{run\PYZus{}planet\PYZus{}simulation} \PY{p}{(}\PY{n}{earth\PYZus{}position\PYZus{}x}\PY{p}{,}\PY{n}{earth\PYZus{}position\PYZus{}y}\PY{p}{,} \PY{n}{earth\PYZus{}velocity\PYZus{}x}\PY{p}{,} \PY{n}{earth\PYZus{}velocity\PYZus{}y}\PY{p}{,}\PY{n}{earth\PYZus{}mass}\PY{p}{,}\PY{n}{num\PYZus{}years}\PY{p}{)}\PY{p}{:}
             \PY{l+s+sd}{\PYZdq{}\PYZdq{}\PYZdq{}This function runs a simulation of a two body planetary system. }
         \PY{l+s+sd}{    This simulation assumes that the sun in the planetary system has the mass of earth\PYZsq{}s sun and is at a fixed location in space (0,0) for conveianance}
         \PY{l+s+sd}{    the planet\PYZsq{}s position velocity and mass are supplied to better inform the simulation.}
         \PY{l+s+sd}{    The number of years dictates how many years the simulation should be run for.}
         \PY{l+s+sd}{    Max\PYZus{}step\PYZus{}size is the maximum step size in seconds}
         \PY{l+s+sd}{    The function returns the trajectory of the supplied planet.\PYZdq{}\PYZdq{}\PYZdq{}}
             \PY{c+c1}{\PYZsh{} creates the state for the simulation}
             \PY{c+c1}{\PYZsh{}distance in meteres velocity in m/s}
             \PY{n}{st8} \PY{o}{=} \PY{n}{State}\PY{p}{(}\PY{n}{xpos} \PY{o}{=} \PY{n}{earth\PYZus{}position\PYZus{}x}\PY{p}{,} 
                         \PY{n}{ypos} \PY{o}{=} \PY{n}{earth\PYZus{}position\PYZus{}y}\PY{p}{,} 
                         \PY{n}{xvel} \PY{o}{=} \PY{n}{earth\PYZus{}velocity\PYZus{}x}\PY{p}{,} 
                         \PY{n}{yvel} \PY{o}{=} \PY{n}{earth\PYZus{}velocity\PYZus{}y}\PY{p}{)}
             \PY{c+c1}{\PYZsh{}Creates a system which holds the parameters of the simulation}
             \PY{c+c1}{\PYZsh{}G = m\PYZca{}3/(kg*s\PYZca{}2) masses are in kg masses are in kg radi are in meteres velocities are in m/s}
             \PY{n}{sys} \PY{o}{=} \PY{n}{System}\PY{p}{(}\PY{n}{init} \PY{o}{=} \PY{n}{st8}\PY{p}{,}
                          \PY{n}{G} \PY{o}{=} \PY{l+m+mf}{6.67408}\PY{o}{*}\PY{l+m+mi}{10}\PY{o}{*}\PY{o}{*}\PY{p}{(}\PY{o}{\PYZhy{}}\PY{l+m+mi}{11}\PY{p}{)}\PY{p}{,} 
                          \PY{n}{M\PYZus{}earth} \PY{o}{=} \PY{n}{earth\PYZus{}mass}\PY{p}{,} 
                          \PY{n}{M\PYZus{}sun} \PY{o}{=}\PY{l+m+mf}{1.989}\PY{o}{*}\PY{l+m+mi}{10}\PY{o}{*}\PY{o}{*}\PY{l+m+mi}{30}\PY{p}{,} 
                          \PY{n}{R\PYZus{}earth} \PY{o}{=} \PY{l+m+mf}{6.3781}\PY{o}{*}\PY{l+m+mi}{10}\PY{o}{*}\PY{o}{*}\PY{l+m+mi}{6}\PY{p}{,} 
                          \PY{n}{R\PYZus{}sun}\PY{o}{=} \PY{l+m+mf}{6.95700}\PY{o}{*}\PY{l+m+mi}{10}\PY{o}{*}\PY{o}{*}\PY{l+m+mi}{8}\PY{p}{,} 
                          \PY{n}{t\PYZus{}end} \PY{o}{=} \PY{l+m+mf}{365.25}\PY{o}{*}\PY{l+m+mi}{24}\PY{o}{*}\PY{l+m+mi}{60}\PY{o}{*}\PY{l+m+mi}{60}\PY{o}{*}\PY{n}{num\PYZus{}years}\PY{p}{,}
                          \PY{n}{Earth\PYZus{}Radius} \PY{o}{=} \PY{l+m+mi}{6378}\PY{o}{*}\PY{l+m+mi}{10}\PY{o}{*}\PY{o}{*}\PY{l+m+mi}{3}\PY{p}{,}
                          \PY{n}{sun\PYZus{}pos} \PY{o}{=} \PY{n}{Vector}\PY{p}{(}\PY{l+m+mi}{0}\PY{p}{,}\PY{l+m+mi}{0}\PY{p}{)}\PY{p}{)}
             
             \PY{c+c1}{\PYZsh{}runs an ODE solver to run the simulation and restricts the ODE solver to taking steps with a maximum size of one day}
             \PY{n}{results}\PY{p}{,}\PY{n}{details} \PY{o}{=} \PY{n}{run\PYZus{}ode\PYZus{}solver}\PY{p}{(}\PY{n}{sys}\PY{p}{,}\PY{n}{planet\PYZus{}slope\PYZus{}func}\PY{p}{,} \PY{n}{max\PYZus{}step}\PY{o}{=}\PY{l+m+mi}{60}\PY{o}{*}\PY{l+m+mi}{60}\PY{o}{*}\PY{l+m+mi}{24}\PY{p}{)}
             
             \PY{c+c1}{\PYZsh{}returns the results of the solver}
             \PY{k}{return} \PY{n}{results}
\end{Verbatim}


    This function simulates a change in momentum of the earth as a result of
the earth firing a projectile. This function performs the conservation
of momentum calculations then it runs a simulation of the earth in orbit
around the sun with the earth's updated mass and velocity as a result of
the change in momentum. This function returns a time frame detailing the
earth's position and velocity at various points in time after the shot
has been fired.

    \begin{Verbatim}[commandchars=\\\{\}]
{\color{incolor}In [{\color{incolor}11}]:} \PY{k}{def} \PY{n+nf}{run\PYZus{}simulation\PYZus{}after\PYZus{}cannon} \PY{p}{(}\PY{n}{Mass\PYZus{}shot}\PY{p}{,} \PY{n}{Shot\PYZus{}velocity}\PY{p}{,} \PY{n}{number\PYZus{}years\PYZus{}after\PYZus{}shot}\PY{p}{,} \PY{n}{orbital\PYZus{}results}\PY{p}{)}\PY{p}{:}
             \PY{l+s+sd}{\PYZdq{}\PYZdq{}\PYZdq{}This code runs a simulation of the earth firing a projectile of a given mass and velocity.}
         \PY{l+s+sd}{    This code is intended to be run after another simulation of normal orbital motion which will inform the starting position of the earth before the projectile is fired.}
         \PY{l+s+sd}{    Masses are in kg}
         \PY{l+s+sd}{    Velocities are in m/s}
         \PY{l+s+sd}{    number of years after shot is a number whic dictates how long to run the simulation of the earth\PYZsq{}s orbital motion after the projectile is fired.}
         \PY{l+s+sd}{    This function returns a time frame containing the x and y poition and velocity of the earth throughout the simulation\PYZdq{}\PYZdq{}\PYZdq{}}
             \PY{c+c1}{\PYZsh{}gets a vector of the velocity of earth based on the results of the original simulation}
             \PY{n}{orbital\PYZus{}velocity} \PY{o}{=} \PY{n}{Vector}\PY{p}{(}\PY{n}{get\PYZus{}last\PYZus{}value}\PY{p}{(}\PY{n}{orbital\PYZus{}results}\PY{o}{.}\PY{n}{xvel}\PY{p}{)}\PY{p}{,}\PY{n}{get\PYZus{}last\PYZus{}value}\PY{p}{(}\PY{n}{orbital\PYZus{}results}\PY{o}{.}\PY{n}{yvel}\PY{p}{)}\PY{p}{)}
             
             \PY{c+c1}{\PYZsh{}fires the railgun}
             \PY{n}{post\PYZus{}cannon\PYZus{}velocity}\PY{p}{,} \PY{n}{post\PYZus{}cannon\PYZus{}mass} \PY{o}{=} \PY{n}{fire\PYZus{}cannon}\PY{p}{(}\PY{n}{orbital\PYZus{}velocity}\PY{p}{,}\PY{l+m+mf}{5.972}\PY{o}{*}\PY{l+m+mi}{10}\PY{o}{*}\PY{o}{*}\PY{l+m+mi}{24}\PY{p}{,} \PY{n}{Mass\PYZus{}shot}\PY{p}{,} \PY{n}{Shot\PYZus{}velocity}\PY{p}{)}
             
             \PY{c+c1}{\PYZsh{}runs a simulation of the earth and the sun with the earth\PYZsq{}s updated velocity allowing the ODE solver to take any step size}
             \PY{n}{escape\PYZus{}results} \PY{o}{=} \PY{n}{run\PYZus{}planet\PYZus{}simulation}\PY{p}{(}\PY{n}{get\PYZus{}last\PYZus{}value}\PY{p}{(}\PY{n}{orbital\PYZus{}results}\PY{o}{.}\PY{n}{xpos}\PY{p}{)}\PY{p}{,}\PY{n}{get\PYZus{}last\PYZus{}value}\PY{p}{(}\PY{n}{orbital\PYZus{}results}\PY{o}{.}\PY{n}{ypos}\PY{p}{)}\PY{p}{,}\PY{n}{post\PYZus{}cannon\PYZus{}velocity}\PY{o}{.}\PY{n}{x}\PY{p}{,}\PY{n}{post\PYZus{}cannon\PYZus{}velocity}\PY{o}{.}\PY{n}{y}\PY{p}{,}\PY{n}{post\PYZus{}cannon\PYZus{}mass}\PY{p}{,}\PY{n}{number\PYZus{}years\PYZus{}after\PYZus{}shot}\PY{p}{)}
             
             \PY{c+c1}{\PYZsh{}escape\PYZus{}results = run\PYZus{}planet\PYZus{}simulation(get\PYZus{}last\PYZus{}value(orbital\PYZus{}results.xpos),get\PYZus{}last\PYZus{}value(orbital\PYZus{}results.ypos),get\PYZus{}last\PYZus{}value(orbital\PYZus{}results.xvel),get\PYZus{}last\PYZus{}value(orbital\PYZus{}results.yvel),5.972*10**24,number\PYZus{}years\PYZus{}after\PYZus{}shot)}
             
             \PY{c+c1}{\PYZsh{}returns the results}
             \PY{k}{return} \PY{n}{escape\PYZus{}results}
\end{Verbatim}


    The remaining functions expand upon the above functions in more detail.
The in depth explanation can be skipped by proceding to the Results
section below.

    \begin{Verbatim}[commandchars=\\\{\}]
{\color{incolor}In [{\color{incolor}12}]:} \PY{k}{def} \PY{n+nf}{planet\PYZus{}slope\PYZus{}func}\PY{p}{(}\PY{n}{st8}\PY{p}{,}\PY{n}{t}\PY{p}{,}\PY{n}{sys}\PY{p}{)}\PY{p}{:}
             \PY{l+s+sd}{\PYZdq{}\PYZdq{}\PYZdq{}Describes the change in position and velocity in both the x and y plane between every time step.}
         \PY{l+s+sd}{    The function takes in a state which describes the old position of the system}
         \PY{l+s+sd}{    t is the time of the syatem, it is not actually used in calculations however it is a required argument for use with an ODE Solver}
         \PY{l+s+sd}{    sys is a system object which contains system constants\PYZdq{}\PYZdq{}\PYZdq{}}
             
             \PY{c+c1}{\PYZsh{}extracts the values from the state because the ODE solver does not store variables in a state object}
             \PY{n}{xpos}\PY{p}{,}\PY{n}{ypos}\PY{p}{,}\PY{n}{xvel}\PY{p}{,}\PY{n}{yvel}\PY{o}{=} \PY{n}{st8}
             \PY{c+c1}{\PYZsh{}creates a vector which describes the position of the planet}
             \PY{n}{planet\PYZus{}pos} \PY{o}{=} \PY{n}{Vector}\PY{p}{(}\PY{n}{xpos}\PY{p}{,}\PY{n}{ypos}\PY{p}{)}
             
             \PY{c+c1}{\PYZsh{}describes a system of differential equations which explain the changes in the system}
             \PY{c+c1}{\PYZsh{}update the velocity in each direction}
             \PY{n}{dxveldt} \PY{o}{=} \PY{n}{get\PYZus{}grav\PYZus{}acceleration}\PY{p}{(}\PY{n}{sys}\PY{o}{.}\PY{n}{G}\PY{p}{,} \PY{n}{sys}\PY{o}{.}\PY{n}{M\PYZus{}sun}\PY{p}{,} \PY{n}{sys}\PY{o}{.}\PY{n}{M\PYZus{}earth}\PY{p}{,} \PY{n}{planet\PYZus{}pos}\PY{p}{,}\PY{n}{sys}\PY{o}{.}\PY{n}{sun\PYZus{}pos}\PY{p}{)}\PY{o}{.}\PY{n}{x}
             \PY{n}{dyveldt} \PY{o}{=} \PY{n}{get\PYZus{}grav\PYZus{}acceleration}\PY{p}{(}\PY{n}{sys}\PY{o}{.}\PY{n}{G}\PY{p}{,} \PY{n}{sys}\PY{o}{.}\PY{n}{M\PYZus{}sun}\PY{p}{,} \PY{n}{sys}\PY{o}{.}\PY{n}{M\PYZus{}earth}\PY{p}{,} \PY{n}{planet\PYZus{}pos}\PY{p}{,}\PY{n}{sys}\PY{o}{.}\PY{n}{sun\PYZus{}pos}\PY{p}{)}\PY{o}{.}\PY{n}{y}
             
             \PY{c+c1}{\PYZsh{}update the position in each direction}
             \PY{n}{dxposdt} \PY{o}{=} \PY{n}{xvel}
             \PY{n}{dyposdt} \PY{o}{=} \PY{n}{yvel}
             
             \PY{c+c1}{\PYZsh{}returns the change in the state variables}
             \PY{k}{return} \PY{n}{dxposdt}\PY{p}{,}\PY{n}{dyposdt}\PY{p}{,}\PY{n}{dxveldt}\PY{p}{,}\PY{n}{dyveldt}
\end{Verbatim}


    The following function performs the momentum conservation calculations
on the earth. This function returns an updated mass and velocity of the
earth based on the change in momentum from the firing of a projecile
with a given mass and velocity.

    \begin{Verbatim}[commandchars=\\\{\}]
{\color{incolor}In [{\color{incolor}13}]:} \PY{k}{def} \PY{n+nf}{fire\PYZus{}cannon}\PY{p}{(}\PY{n}{planet\PYZus{}velocity\PYZus{}vector}\PY{p}{,} \PY{n}{Mass\PYZus{}earth}\PY{p}{,} \PY{n}{Mass\PYZus{}shot}\PY{p}{,} \PY{n}{Shot\PYZus{}speed}\PY{p}{)}\PY{p}{:}
             \PY{l+s+sd}{\PYZdq{}\PYZdq{}\PYZdq{}Runs a function which updates the velocity and mass of a planet assuming that it is accelerated via mass ejection.}
         \PY{l+s+sd}{    Calculations are based on conservation of momentum.}
         \PY{l+s+sd}{    Masses are in kilograms}
         \PY{l+s+sd}{    Speeds are in m/s}
         \PY{l+s+sd}{    The velocity vector describes the last known velocity of the planet.}
         \PY{l+s+sd}{    MODELING ASSUMPTION CONSTRAINT:  This model assumes that the earth exerts no gravitational pull on the projectile.}
         \PY{l+s+sd}{    This means that projectiles fired at less than their escape velocity will not be accurate to the real world where they would return to the earth and recombine with it.\PYZdq{}\PYZdq{}\PYZdq{}}
             \PY{k}{if} \PY{n}{Shot\PYZus{}speed} \PY{o}{!=} \PY{l+m+mi}{0}\PY{p}{:}
                 \PY{c+c1}{\PYZsh{}calculates the magnitude of the new velocity of the planet}
                 \PY{n}{velocity\PYZus{}vector\PYZus{}magnitude} \PY{o}{=} \PY{p}{(}\PY{n}{Mass\PYZus{}earth}\PY{o}{*}\PY{n}{planet\PYZus{}velocity\PYZus{}vector}\PY{o}{.}\PY{n}{mag} \PY{o}{+} \PY{n}{Mass\PYZus{}shot}\PY{o}{*}\PY{n}{Shot\PYZus{}speed}\PY{p}{)}\PY{o}{/}\PY{p}{(}\PY{n}{Mass\PYZus{}earth}\PY{o}{\PYZhy{}}\PY{n}{Mass\PYZus{}shot}\PY{p}{)} 
                 \PY{c+c1}{\PYZsh{}multiplies the magnitude by a unit vector in the direction of the last known velocity to create a velocity vector in the proper direction}
                 \PY{n}{planet\PYZus{}velocity\PYZus{}vector} \PY{o}{=} \PY{p}{(}\PY{n}{planet\PYZus{}velocity\PYZus{}vector}\PY{o}{/}\PY{n}{planet\PYZus{}velocity\PYZus{}vector}\PY{o}{.}\PY{n}{mag}\PY{p}{)}\PY{o}{*}\PY{n}{velocity\PYZus{}vector\PYZus{}magnitude}
          
                 \PY{c+c1}{\PYZsh{}updates the mass of the earth to account for the fact that the shot removed mass from the earth}
                 \PY{n}{Mass\PYZus{}earth} \PY{o}{=} \PY{n}{Mass\PYZus{}earth} \PY{o}{\PYZhy{}} \PY{n}{Mass\PYZus{}shot}
             
             \PY{c+c1}{\PYZsh{}retuns the new velocity as a vector and the new mass of the earth}
             \PY{k}{return} \PY{n}{planet\PYZus{}velocity\PYZus{}vector}\PY{p}{,} \PY{n}{Mass\PYZus{}earth}
\end{Verbatim}


    This function finds the magnitude of the gravitational acceleration on a
planet by the sun using F = ma. The function returns the acceleration in
terms of m/s\^{}2.

    \begin{Verbatim}[commandchars=\\\{\}]
{\color{incolor}In [{\color{incolor}14}]:} \PY{k}{def} \PY{n+nf}{get\PYZus{}grav\PYZus{}acceleration}\PY{p}{(}\PY{n}{G}\PY{p}{,}\PY{n}{M}\PY{p}{,}\PY{n}{m}\PY{p}{,}\PY{n}{planet\PYZus{}vector}\PY{p}{,}\PY{n}{sun\PYZus{}vector}\PY{p}{)}\PY{p}{:}
             \PY{l+s+sd}{\PYZdq{}\PYZdq{}\PYZdq{}Calculates acceleration due to gravity with F = ma}
         \PY{l+s+sd}{    F in N}
         \PY{l+s+sd}{    m in kg}
         \PY{l+s+sd}{    a in m/s\PYZca{}2\PYZdq{}\PYZdq{}\PYZdq{}}
             \PY{c+c1}{\PYZsh{}gets the force and divides by the orbiting body to get the acceleration of the orbing body}
             \PY{k}{return} \PY{n}{get\PYZus{}gravitational\PYZus{}force}\PY{p}{(}\PY{n}{G}\PY{p}{,}\PY{n}{M}\PY{p}{,}\PY{n}{m}\PY{p}{,}\PY{n}{planet\PYZus{}vector}\PY{p}{,}\PY{n}{sun\PYZus{}vector}\PY{p}{)}\PY{o}{/}\PY{n}{m}
\end{Verbatim}


    The following function applies the law of universal gravitation to find
the magnitude of the gravitational force for a planet orbiting a star.
The function retunrs a force in N which describes the gravitational
force.

    \begin{Verbatim}[commandchars=\\\{\}]
{\color{incolor}In [{\color{incolor}15}]:} \PY{k}{def} \PY{n+nf}{get\PYZus{}gravitational\PYZus{}force}\PY{p}{(}\PY{n}{G}\PY{p}{,}\PY{n}{M}\PY{p}{,}\PY{n}{m}\PY{p}{,}\PY{n}{planet\PYZus{}vector}\PY{p}{,} \PY{n}{sun\PYZus{}vector}\PY{p}{)}\PY{p}{:}
             \PY{l+s+sd}{\PYZdq{}\PYZdq{}\PYZdq{}Applies law of universal gravitation and returns the force of gravity as a vector towards the center of the circle in N\PYZdq{}\PYZdq{}\PYZdq{}}
             \PY{c+c1}{\PYZsh{}calculates the vector pointing from the planet to the sun}
             \PY{n}{planet\PYZus{}to\PYZus{}sun\PYZus{}vector} \PY{o}{=} \PY{n}{sun\PYZus{}vector}\PY{o}{\PYZhy{}}\PY{n}{planet\PYZus{}vector}
             \PY{c+c1}{\PYZsh{}takes the direction of the previous vector}
             \PY{n}{Direction\PYZus{}vector} \PY{o}{=} \PY{n}{planet\PYZus{}to\PYZus{}sun\PYZus{}vector}\PY{o}{/}\PY{n}{planet\PYZus{}to\PYZus{}sun\PYZus{}vector}\PY{o}{.}\PY{n}{mag}
             \PY{c+c1}{\PYZsh{}calculates the acceleration of the planet}
             \PY{n}{acceleration\PYZus{}mag} \PY{o}{=} \PY{p}{(}\PY{n}{G}\PY{o}{*}\PY{n}{M}\PY{o}{*}\PY{n}{m}\PY{p}{)}\PY{o}{/}\PY{p}{(}\PY{n}{planet\PYZus{}vector}\PY{o}{.}\PY{n}{mag}\PY{o}{*}\PY{o}{*}\PY{l+m+mi}{2}\PY{p}{)}
             \PY{c+c1}{\PYZsh{}returns a vector which has the magnitude of the acceleration and a direction towards the star}
             \PY{k}{return} \PY{n}{Direction\PYZus{}vector} \PY{o}{*} \PY{n}{acceleration\PYZus{}mag}
\end{Verbatim}


    \hypertarget{results}{%
\section{Results}\label{results}}

    The following code describes the inital state of the earth. values have
been chosen to place the earth at a distance of one astronomical unit
(AU) away from the sun (the average distance from the earth to the sun),
moving at a velocity which corrasponds to the observed velocity
withwhich the earth orbits the sun (assuming a nearly circular orbit and
a distance from the sun of 1 AU), and the measured mass of the earth.
The simulation is set to run for one year to demonstrate earth in stable
orbit before the railgun is fired.

    \begin{Verbatim}[commandchars=\\\{\}]
{\color{incolor}In [{\color{incolor}17}]:} \PY{c+c1}{\PYZsh{}Describes the inital position, in vector format, of the earth (m)}
         \PY{n}{inital\PYZus{}earth\PYZus{}x\PYZus{}position} \PY{o}{=} \PY{l+m+mi}{149597900000}
         \PY{n}{inital\PYZus{}earth\PYZus{}y\PYZus{}position} \PY{o}{=} \PY{l+m+mi}{0}
         \PY{c+c1}{\PYZsh{}Describes the inital velocity, in vector format, of the earth (m/s)}
         \PY{n}{inital\PYZus{}earth\PYZus{}x\PYZus{}velocity} \PY{o}{=} \PY{l+m+mi}{0}
         \PY{n}{inital\PYZus{}earth\PYZus{}y\PYZus{}velocity} \PY{o}{=} \PY{l+m+mi}{29785}
         \PY{c+c1}{\PYZsh{}Describes the inital mass of the earth (kg)}
         \PY{n}{inital\PYZus{}earth\PYZus{}mass} \PY{o}{=} \PY{l+m+mf}{5.972}\PY{o}{*}\PY{l+m+mi}{10}\PY{o}{*}\PY{o}{*}\PY{l+m+mi}{24}
         \PY{c+c1}{\PYZsh{}Describes the length of the stable orbit simulation (years)}
         \PY{n}{stable\PYZus{}orbit\PYZus{}simulation\PYZus{}runtime} \PY{o}{=} \PY{l+m+mi}{1}
         \PY{k+kc}{None}
\end{Verbatim}


    The following code runs a simulation whic sweeps acoss masses of
projectiles ranging from 1000 kg (about the mass of the average car) to
1*10\^{}24 kg (roughly one fifth the mass of the earth)

    \begin{Verbatim}[commandchars=\\\{\}]
{\color{incolor}In [{\color{incolor}18}]:} \PY{n}{orbital\PYZus{}results} \PY{o}{=} \PY{n}{run\PYZus{}planet\PYZus{}simulation}\PY{p}{(}\PY{n}{inital\PYZus{}earth\PYZus{}x\PYZus{}position}\PY{p}{,}\PY{n}{inital\PYZus{}earth\PYZus{}y\PYZus{}position}\PY{p}{,} \PY{n}{inital\PYZus{}earth\PYZus{}x\PYZus{}velocity}\PY{p}{,} \PY{n}{inital\PYZus{}earth\PYZus{}y\PYZus{}velocity}\PY{p}{,}\PY{n}{inital\PYZus{}earth\PYZus{}mass}\PY{p}{,}\PY{n}{stable\PYZus{}orbit\PYZus{}simulation\PYZus{}runtime}\PY{p}{)} 
         \PY{n}{plt}\PY{o}{.}\PY{n}{figure}\PY{p}{(}\PY{l+m+mi}{1}\PY{p}{)}
         \PY{n}{plot}\PY{p}{(}\PY{n}{orbital\PYZus{}results}\PY{o}{.}\PY{n}{xpos}\PY{p}{,} \PY{n}{orbital\PYZus{}results}\PY{o}{.}\PY{n}{ypos}\PY{p}{)}
         \PY{c+c1}{\PYZsh{} Inputs as parameters}
         
         \PY{n}{possible\PYZus{}masses} \PY{o}{=} \PY{n}{np}\PY{o}{.}\PY{n}{logspace}\PY{p}{(}\PY{l+m+mi}{3}\PY{p}{,} \PY{l+m+mi}{24}\PY{p}{,}\PY{l+m+mi}{5}\PY{p}{)}
         \PY{k}{for} \PY{n}{mass} \PY{o+ow}{in} \PY{n}{possible\PYZus{}masses}\PY{p}{:}
             \PY{n}{v\PYZus{}s} \PY{o}{=} \PY{l+m+mf}{1e15} \PY{o}{*} \PY{n}{V}
             \PY{n}{shot\PYZus{}results} \PY{o}{=} \PY{n}{launch\PYZus{}simulation}\PY{p}{(}\PY{n}{mass}\PY{p}{,} \PY{n}{v\PYZus{}s}\PY{p}{,} \PY{n}{params\PYZus{}railgun}\PY{p}{,} \PY{l+m+mi}{1}\PY{p}{,} \PY{n}{orbital\PYZus{}results}\PY{p}{)}
             \PY{n}{plot}\PY{p}{(}\PY{n}{shot\PYZus{}results}\PY{o}{.}\PY{n}{xpos}\PY{p}{,}\PY{n}{shot\PYZus{}results}\PY{o}{.}\PY{n}{ypos}\PY{p}{,} \PY{n}{label} \PY{o}{=} \PY{n+nb}{str}\PY{p}{(}\PY{n}{mass}\PY{p}{)}\PY{o}{+} \PY{l+s+s2}{\PYZdq{}}\PY{l+s+s2}{ kg}\PY{l+s+s2}{\PYZdq{}}\PY{p}{)}
         \PY{n}{plt}\PY{o}{.}\PY{n}{xlabel}\PY{p}{(}\PY{l+s+s1}{\PYZsq{}}\PY{l+s+s1}{Distance from the Sun along orthoganal axes(m)}\PY{l+s+s1}{\PYZsq{}}\PY{p}{)}
         \PY{n}{plt}\PY{o}{.}\PY{n}{ylabel}\PY{p}{(}\PY{l+s+s1}{\PYZsq{}}\PY{l+s+s1}{Distance from the Sun along orthoganal axes (m)}\PY{l+s+s1}{\PYZsq{}}\PY{p}{)}
         \PY{n}{plt}\PY{o}{.}\PY{n}{title}\PY{p}{(}\PY{l+s+s1}{\PYZsq{}}\PY{l+s+s1}{Orbit of the Earth before vs. after firing the railgun}\PY{l+s+s1}{\PYZsq{}}\PY{p}{)}
         \PY{k+kc}{None}
\end{Verbatim}


    \begin{center}
    \adjustimage{max size={0.9\linewidth}{0.9\paperheight}}{output_32_0.png}
    \end{center}
    { \hspace*{\fill} \\}
    
    \begin{Verbatim}[commandchars=\\\{\}]
{\color{incolor}In [{\color{incolor}19}]:} \PY{n}{fin\PYZus{}v} \PY{o}{=} \PY{n}{pd}\PY{o}{.}\PY{n}{Series}\PY{p}{(}\PY{n}{index} \PY{o}{=} \PY{n}{possible\PYZus{}masses}\PY{p}{)}
         \PY{k}{for} \PY{n}{mass} \PY{o+ow}{in} \PY{n}{possible\PYZus{}masses}\PY{p}{:}
             \PY{n}{v\PYZus{}s} \PY{o}{=} \PY{l+m+mf}{1e27} \PY{o}{*} \PY{n}{V}
             \PY{n}{x}\PY{p}{,} \PY{n}{v} \PY{o}{=} \PY{n}{sim\PYZus{}railgun}\PY{p}{(}\PY{n}{mass}\PY{p}{,} \PY{n}{v\PYZus{}s}\PY{p}{,} \PY{n}{params\PYZus{}railgun}\PY{p}{)}
             \PY{n}{fin\PYZus{}v}\PY{p}{[}\PY{n}{mass}\PY{p}{]} \PY{o}{=} \PY{n}{get\PYZus{}last\PYZus{}value}\PY{p}{(}\PY{n}{v}\PY{p}{)}
         \PY{n}{fin\PYZus{}v}
\end{Verbatim}


\begin{Verbatim}[commandchars=\\\{\}]
{\color{outcolor}Out[{\color{outcolor}19}]:} 1.000000e+03    0.000000e+00
         1.778279e+08    0.000000e+00
         3.162278e+13    0.000000e+00
         5.623413e+18    0.000000e+00
         1.000000e+24    2.880313e+16
         dtype: float64
\end{Verbatim}
            
    \begin{Verbatim}[commandchars=\\\{\}]
{\color{incolor}In [{\color{incolor}20}]:} \PY{l+m+mf}{1e27}
\end{Verbatim}


\begin{Verbatim}[commandchars=\\\{\}]
{\color{outcolor}Out[{\color{outcolor}20}]:} 1e+27
\end{Verbatim}
            

    % Add a bibliography block to the postdoc
    
    
    
    \end{document}
