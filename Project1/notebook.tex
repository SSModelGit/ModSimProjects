
% Default to the notebook output style

    


% Inherit from the specified cell style.




    
\documentclass[11pt]{article}

    
    
    \usepackage[T1]{fontenc}
    % Nicer default font (+ math font) than Computer Modern for most use cases
    \usepackage{mathpazo}

    % Basic figure setup, for now with no caption control since it's done
    % automatically by Pandoc (which extracts ![](path) syntax from Markdown).
    \usepackage{graphicx}
    % We will generate all images so they have a width \maxwidth. This means
    % that they will get their normal width if they fit onto the page, but
    % are scaled down if they would overflow the margins.
    \makeatletter
    \def\maxwidth{\ifdim\Gin@nat@width>\linewidth\linewidth
    \else\Gin@nat@width\fi}
    \makeatother
    \let\Oldincludegraphics\includegraphics
    % Set max figure width to be 80% of text width, for now hardcoded.
    \renewcommand{\includegraphics}[1]{\Oldincludegraphics[width=.8\maxwidth]{#1}}
    % Ensure that by default, figures have no caption (until we provide a
    % proper Figure object with a Caption API and a way to capture that
    % in the conversion process - todo).
    \usepackage{caption}
    \DeclareCaptionLabelFormat{nolabel}{}
    \captionsetup{labelformat=nolabel}

    \usepackage{adjustbox} % Used to constrain images to a maximum size 
    \usepackage{xcolor} % Allow colors to be defined
    \usepackage{enumerate} % Needed for markdown enumerations to work
    \usepackage{geometry} % Used to adjust the document margins
    \usepackage{amsmath} % Equations
    \usepackage{amssymb} % Equations
    \usepackage{textcomp} % defines textquotesingle
    % Hack from http://tex.stackexchange.com/a/47451/13684:
    \AtBeginDocument{%
        \def\PYZsq{\textquotesingle}% Upright quotes in Pygmentized code
    }
    \usepackage{upquote} % Upright quotes for verbatim code
    \usepackage{eurosym} % defines \euro
    \usepackage[mathletters]{ucs} % Extended unicode (utf-8) support
    \usepackage[utf8x]{inputenc} % Allow utf-8 characters in the tex document
    \usepackage{fancyvrb} % verbatim replacement that allows latex
    \usepackage{grffile} % extends the file name processing of package graphics 
                         % to support a larger range 
    % The hyperref package gives us a pdf with properly built
    % internal navigation ('pdf bookmarks' for the table of contents,
    % internal cross-reference links, web links for URLs, etc.)
    \usepackage{hyperref}
    \usepackage{longtable} % longtable support required by pandoc >1.10
    \usepackage{booktabs}  % table support for pandoc > 1.12.2
    \usepackage[inline]{enumitem} % IRkernel/repr support (it uses the enumerate* environment)
    \usepackage[normalem]{ulem} % ulem is needed to support strikethroughs (\sout)
                                % normalem makes italics be italics, not underlines
    

    
    
    % Colors for the hyperref package
    \definecolor{urlcolor}{rgb}{0,.145,.698}
    \definecolor{linkcolor}{rgb}{.71,0.21,0.01}
    \definecolor{citecolor}{rgb}{.12,.54,.11}

    % ANSI colors
    \definecolor{ansi-black}{HTML}{3E424D}
    \definecolor{ansi-black-intense}{HTML}{282C36}
    \definecolor{ansi-red}{HTML}{E75C58}
    \definecolor{ansi-red-intense}{HTML}{B22B31}
    \definecolor{ansi-green}{HTML}{00A250}
    \definecolor{ansi-green-intense}{HTML}{007427}
    \definecolor{ansi-yellow}{HTML}{DDB62B}
    \definecolor{ansi-yellow-intense}{HTML}{B27D12}
    \definecolor{ansi-blue}{HTML}{208FFB}
    \definecolor{ansi-blue-intense}{HTML}{0065CA}
    \definecolor{ansi-magenta}{HTML}{D160C4}
    \definecolor{ansi-magenta-intense}{HTML}{A03196}
    \definecolor{ansi-cyan}{HTML}{60C6C8}
    \definecolor{ansi-cyan-intense}{HTML}{258F8F}
    \definecolor{ansi-white}{HTML}{C5C1B4}
    \definecolor{ansi-white-intense}{HTML}{A1A6B2}

    % commands and environments needed by pandoc snippets
    % extracted from the output of `pandoc -s`
    \providecommand{\tightlist}{%
      \setlength{\itemsep}{0pt}\setlength{\parskip}{0pt}}
    \DefineVerbatimEnvironment{Highlighting}{Verbatim}{commandchars=\\\{\}}
    % Add ',fontsize=\small' for more characters per line
    \newenvironment{Shaded}{}{}
    \newcommand{\KeywordTok}[1]{\textcolor[rgb]{0.00,0.44,0.13}{\textbf{{#1}}}}
    \newcommand{\DataTypeTok}[1]{\textcolor[rgb]{0.56,0.13,0.00}{{#1}}}
    \newcommand{\DecValTok}[1]{\textcolor[rgb]{0.25,0.63,0.44}{{#1}}}
    \newcommand{\BaseNTok}[1]{\textcolor[rgb]{0.25,0.63,0.44}{{#1}}}
    \newcommand{\FloatTok}[1]{\textcolor[rgb]{0.25,0.63,0.44}{{#1}}}
    \newcommand{\CharTok}[1]{\textcolor[rgb]{0.25,0.44,0.63}{{#1}}}
    \newcommand{\StringTok}[1]{\textcolor[rgb]{0.25,0.44,0.63}{{#1}}}
    \newcommand{\CommentTok}[1]{\textcolor[rgb]{0.38,0.63,0.69}{\textit{{#1}}}}
    \newcommand{\OtherTok}[1]{\textcolor[rgb]{0.00,0.44,0.13}{{#1}}}
    \newcommand{\AlertTok}[1]{\textcolor[rgb]{1.00,0.00,0.00}{\textbf{{#1}}}}
    \newcommand{\FunctionTok}[1]{\textcolor[rgb]{0.02,0.16,0.49}{{#1}}}
    \newcommand{\RegionMarkerTok}[1]{{#1}}
    \newcommand{\ErrorTok}[1]{\textcolor[rgb]{1.00,0.00,0.00}{\textbf{{#1}}}}
    \newcommand{\NormalTok}[1]{{#1}}
    
    % Additional commands for more recent versions of Pandoc
    \newcommand{\ConstantTok}[1]{\textcolor[rgb]{0.53,0.00,0.00}{{#1}}}
    \newcommand{\SpecialCharTok}[1]{\textcolor[rgb]{0.25,0.44,0.63}{{#1}}}
    \newcommand{\VerbatimStringTok}[1]{\textcolor[rgb]{0.25,0.44,0.63}{{#1}}}
    \newcommand{\SpecialStringTok}[1]{\textcolor[rgb]{0.73,0.40,0.53}{{#1}}}
    \newcommand{\ImportTok}[1]{{#1}}
    \newcommand{\DocumentationTok}[1]{\textcolor[rgb]{0.73,0.13,0.13}{\textit{{#1}}}}
    \newcommand{\AnnotationTok}[1]{\textcolor[rgb]{0.38,0.63,0.69}{\textbf{\textit{{#1}}}}}
    \newcommand{\CommentVarTok}[1]{\textcolor[rgb]{0.38,0.63,0.69}{\textbf{\textit{{#1}}}}}
    \newcommand{\VariableTok}[1]{\textcolor[rgb]{0.10,0.09,0.49}{{#1}}}
    \newcommand{\ControlFlowTok}[1]{\textcolor[rgb]{0.00,0.44,0.13}{\textbf{{#1}}}}
    \newcommand{\OperatorTok}[1]{\textcolor[rgb]{0.40,0.40,0.40}{{#1}}}
    \newcommand{\BuiltInTok}[1]{{#1}}
    \newcommand{\ExtensionTok}[1]{{#1}}
    \newcommand{\PreprocessorTok}[1]{\textcolor[rgb]{0.74,0.48,0.00}{{#1}}}
    \newcommand{\AttributeTok}[1]{\textcolor[rgb]{0.49,0.56,0.16}{{#1}}}
    \newcommand{\InformationTok}[1]{\textcolor[rgb]{0.38,0.63,0.69}{\textbf{\textit{{#1}}}}}
    \newcommand{\WarningTok}[1]{\textcolor[rgb]{0.38,0.63,0.69}{\textbf{\textit{{#1}}}}}
    
    
    % Define a nice break command that doesn't care if a line doesn't already
    % exist.
    \def\br{\hspace*{\fill} \\* }
    % Math Jax compatability definitions
    \def\gt{>}
    \def\lt{<}
    % Document parameters
    \title{AgeDemographicModelingWork}
    
    
    

    % Pygments definitions
    
\makeatletter
\def\PY@reset{\let\PY@it=\relax \let\PY@bf=\relax%
    \let\PY@ul=\relax \let\PY@tc=\relax%
    \let\PY@bc=\relax \let\PY@ff=\relax}
\def\PY@tok#1{\csname PY@tok@#1\endcsname}
\def\PY@toks#1+{\ifx\relax#1\empty\else%
    \PY@tok{#1}\expandafter\PY@toks\fi}
\def\PY@do#1{\PY@bc{\PY@tc{\PY@ul{%
    \PY@it{\PY@bf{\PY@ff{#1}}}}}}}
\def\PY#1#2{\PY@reset\PY@toks#1+\relax+\PY@do{#2}}

\expandafter\def\csname PY@tok@w\endcsname{\def\PY@tc##1{\textcolor[rgb]{0.73,0.73,0.73}{##1}}}
\expandafter\def\csname PY@tok@c\endcsname{\let\PY@it=\textit\def\PY@tc##1{\textcolor[rgb]{0.25,0.50,0.50}{##1}}}
\expandafter\def\csname PY@tok@cp\endcsname{\def\PY@tc##1{\textcolor[rgb]{0.74,0.48,0.00}{##1}}}
\expandafter\def\csname PY@tok@k\endcsname{\let\PY@bf=\textbf\def\PY@tc##1{\textcolor[rgb]{0.00,0.50,0.00}{##1}}}
\expandafter\def\csname PY@tok@kp\endcsname{\def\PY@tc##1{\textcolor[rgb]{0.00,0.50,0.00}{##1}}}
\expandafter\def\csname PY@tok@kt\endcsname{\def\PY@tc##1{\textcolor[rgb]{0.69,0.00,0.25}{##1}}}
\expandafter\def\csname PY@tok@o\endcsname{\def\PY@tc##1{\textcolor[rgb]{0.40,0.40,0.40}{##1}}}
\expandafter\def\csname PY@tok@ow\endcsname{\let\PY@bf=\textbf\def\PY@tc##1{\textcolor[rgb]{0.67,0.13,1.00}{##1}}}
\expandafter\def\csname PY@tok@nb\endcsname{\def\PY@tc##1{\textcolor[rgb]{0.00,0.50,0.00}{##1}}}
\expandafter\def\csname PY@tok@nf\endcsname{\def\PY@tc##1{\textcolor[rgb]{0.00,0.00,1.00}{##1}}}
\expandafter\def\csname PY@tok@nc\endcsname{\let\PY@bf=\textbf\def\PY@tc##1{\textcolor[rgb]{0.00,0.00,1.00}{##1}}}
\expandafter\def\csname PY@tok@nn\endcsname{\let\PY@bf=\textbf\def\PY@tc##1{\textcolor[rgb]{0.00,0.00,1.00}{##1}}}
\expandafter\def\csname PY@tok@ne\endcsname{\let\PY@bf=\textbf\def\PY@tc##1{\textcolor[rgb]{0.82,0.25,0.23}{##1}}}
\expandafter\def\csname PY@tok@nv\endcsname{\def\PY@tc##1{\textcolor[rgb]{0.10,0.09,0.49}{##1}}}
\expandafter\def\csname PY@tok@no\endcsname{\def\PY@tc##1{\textcolor[rgb]{0.53,0.00,0.00}{##1}}}
\expandafter\def\csname PY@tok@nl\endcsname{\def\PY@tc##1{\textcolor[rgb]{0.63,0.63,0.00}{##1}}}
\expandafter\def\csname PY@tok@ni\endcsname{\let\PY@bf=\textbf\def\PY@tc##1{\textcolor[rgb]{0.60,0.60,0.60}{##1}}}
\expandafter\def\csname PY@tok@na\endcsname{\def\PY@tc##1{\textcolor[rgb]{0.49,0.56,0.16}{##1}}}
\expandafter\def\csname PY@tok@nt\endcsname{\let\PY@bf=\textbf\def\PY@tc##1{\textcolor[rgb]{0.00,0.50,0.00}{##1}}}
\expandafter\def\csname PY@tok@nd\endcsname{\def\PY@tc##1{\textcolor[rgb]{0.67,0.13,1.00}{##1}}}
\expandafter\def\csname PY@tok@s\endcsname{\def\PY@tc##1{\textcolor[rgb]{0.73,0.13,0.13}{##1}}}
\expandafter\def\csname PY@tok@sd\endcsname{\let\PY@it=\textit\def\PY@tc##1{\textcolor[rgb]{0.73,0.13,0.13}{##1}}}
\expandafter\def\csname PY@tok@si\endcsname{\let\PY@bf=\textbf\def\PY@tc##1{\textcolor[rgb]{0.73,0.40,0.53}{##1}}}
\expandafter\def\csname PY@tok@se\endcsname{\let\PY@bf=\textbf\def\PY@tc##1{\textcolor[rgb]{0.73,0.40,0.13}{##1}}}
\expandafter\def\csname PY@tok@sr\endcsname{\def\PY@tc##1{\textcolor[rgb]{0.73,0.40,0.53}{##1}}}
\expandafter\def\csname PY@tok@ss\endcsname{\def\PY@tc##1{\textcolor[rgb]{0.10,0.09,0.49}{##1}}}
\expandafter\def\csname PY@tok@sx\endcsname{\def\PY@tc##1{\textcolor[rgb]{0.00,0.50,0.00}{##1}}}
\expandafter\def\csname PY@tok@m\endcsname{\def\PY@tc##1{\textcolor[rgb]{0.40,0.40,0.40}{##1}}}
\expandafter\def\csname PY@tok@gh\endcsname{\let\PY@bf=\textbf\def\PY@tc##1{\textcolor[rgb]{0.00,0.00,0.50}{##1}}}
\expandafter\def\csname PY@tok@gu\endcsname{\let\PY@bf=\textbf\def\PY@tc##1{\textcolor[rgb]{0.50,0.00,0.50}{##1}}}
\expandafter\def\csname PY@tok@gd\endcsname{\def\PY@tc##1{\textcolor[rgb]{0.63,0.00,0.00}{##1}}}
\expandafter\def\csname PY@tok@gi\endcsname{\def\PY@tc##1{\textcolor[rgb]{0.00,0.63,0.00}{##1}}}
\expandafter\def\csname PY@tok@gr\endcsname{\def\PY@tc##1{\textcolor[rgb]{1.00,0.00,0.00}{##1}}}
\expandafter\def\csname PY@tok@ge\endcsname{\let\PY@it=\textit}
\expandafter\def\csname PY@tok@gs\endcsname{\let\PY@bf=\textbf}
\expandafter\def\csname PY@tok@gp\endcsname{\let\PY@bf=\textbf\def\PY@tc##1{\textcolor[rgb]{0.00,0.00,0.50}{##1}}}
\expandafter\def\csname PY@tok@go\endcsname{\def\PY@tc##1{\textcolor[rgb]{0.53,0.53,0.53}{##1}}}
\expandafter\def\csname PY@tok@gt\endcsname{\def\PY@tc##1{\textcolor[rgb]{0.00,0.27,0.87}{##1}}}
\expandafter\def\csname PY@tok@err\endcsname{\def\PY@bc##1{\setlength{\fboxsep}{0pt}\fcolorbox[rgb]{1.00,0.00,0.00}{1,1,1}{\strut ##1}}}
\expandafter\def\csname PY@tok@kc\endcsname{\let\PY@bf=\textbf\def\PY@tc##1{\textcolor[rgb]{0.00,0.50,0.00}{##1}}}
\expandafter\def\csname PY@tok@kd\endcsname{\let\PY@bf=\textbf\def\PY@tc##1{\textcolor[rgb]{0.00,0.50,0.00}{##1}}}
\expandafter\def\csname PY@tok@kn\endcsname{\let\PY@bf=\textbf\def\PY@tc##1{\textcolor[rgb]{0.00,0.50,0.00}{##1}}}
\expandafter\def\csname PY@tok@kr\endcsname{\let\PY@bf=\textbf\def\PY@tc##1{\textcolor[rgb]{0.00,0.50,0.00}{##1}}}
\expandafter\def\csname PY@tok@bp\endcsname{\def\PY@tc##1{\textcolor[rgb]{0.00,0.50,0.00}{##1}}}
\expandafter\def\csname PY@tok@fm\endcsname{\def\PY@tc##1{\textcolor[rgb]{0.00,0.00,1.00}{##1}}}
\expandafter\def\csname PY@tok@vc\endcsname{\def\PY@tc##1{\textcolor[rgb]{0.10,0.09,0.49}{##1}}}
\expandafter\def\csname PY@tok@vg\endcsname{\def\PY@tc##1{\textcolor[rgb]{0.10,0.09,0.49}{##1}}}
\expandafter\def\csname PY@tok@vi\endcsname{\def\PY@tc##1{\textcolor[rgb]{0.10,0.09,0.49}{##1}}}
\expandafter\def\csname PY@tok@vm\endcsname{\def\PY@tc##1{\textcolor[rgb]{0.10,0.09,0.49}{##1}}}
\expandafter\def\csname PY@tok@sa\endcsname{\def\PY@tc##1{\textcolor[rgb]{0.73,0.13,0.13}{##1}}}
\expandafter\def\csname PY@tok@sb\endcsname{\def\PY@tc##1{\textcolor[rgb]{0.73,0.13,0.13}{##1}}}
\expandafter\def\csname PY@tok@sc\endcsname{\def\PY@tc##1{\textcolor[rgb]{0.73,0.13,0.13}{##1}}}
\expandafter\def\csname PY@tok@dl\endcsname{\def\PY@tc##1{\textcolor[rgb]{0.73,0.13,0.13}{##1}}}
\expandafter\def\csname PY@tok@s2\endcsname{\def\PY@tc##1{\textcolor[rgb]{0.73,0.13,0.13}{##1}}}
\expandafter\def\csname PY@tok@sh\endcsname{\def\PY@tc##1{\textcolor[rgb]{0.73,0.13,0.13}{##1}}}
\expandafter\def\csname PY@tok@s1\endcsname{\def\PY@tc##1{\textcolor[rgb]{0.73,0.13,0.13}{##1}}}
\expandafter\def\csname PY@tok@mb\endcsname{\def\PY@tc##1{\textcolor[rgb]{0.40,0.40,0.40}{##1}}}
\expandafter\def\csname PY@tok@mf\endcsname{\def\PY@tc##1{\textcolor[rgb]{0.40,0.40,0.40}{##1}}}
\expandafter\def\csname PY@tok@mh\endcsname{\def\PY@tc##1{\textcolor[rgb]{0.40,0.40,0.40}{##1}}}
\expandafter\def\csname PY@tok@mi\endcsname{\def\PY@tc##1{\textcolor[rgb]{0.40,0.40,0.40}{##1}}}
\expandafter\def\csname PY@tok@il\endcsname{\def\PY@tc##1{\textcolor[rgb]{0.40,0.40,0.40}{##1}}}
\expandafter\def\csname PY@tok@mo\endcsname{\def\PY@tc##1{\textcolor[rgb]{0.40,0.40,0.40}{##1}}}
\expandafter\def\csname PY@tok@ch\endcsname{\let\PY@it=\textit\def\PY@tc##1{\textcolor[rgb]{0.25,0.50,0.50}{##1}}}
\expandafter\def\csname PY@tok@cm\endcsname{\let\PY@it=\textit\def\PY@tc##1{\textcolor[rgb]{0.25,0.50,0.50}{##1}}}
\expandafter\def\csname PY@tok@cpf\endcsname{\let\PY@it=\textit\def\PY@tc##1{\textcolor[rgb]{0.25,0.50,0.50}{##1}}}
\expandafter\def\csname PY@tok@c1\endcsname{\let\PY@it=\textit\def\PY@tc##1{\textcolor[rgb]{0.25,0.50,0.50}{##1}}}
\expandafter\def\csname PY@tok@cs\endcsname{\let\PY@it=\textit\def\PY@tc##1{\textcolor[rgb]{0.25,0.50,0.50}{##1}}}

\def\PYZbs{\char`\\}
\def\PYZus{\char`\_}
\def\PYZob{\char`\{}
\def\PYZcb{\char`\}}
\def\PYZca{\char`\^}
\def\PYZam{\char`\&}
\def\PYZlt{\char`\<}
\def\PYZgt{\char`\>}
\def\PYZsh{\char`\#}
\def\PYZpc{\char`\%}
\def\PYZdl{\char`\$}
\def\PYZhy{\char`\-}
\def\PYZsq{\char`\'}
\def\PYZdq{\char`\"}
\def\PYZti{\char`\~}
% for compatibility with earlier versions
\def\PYZat{@}
\def\PYZlb{[}
\def\PYZrb{]}
\makeatother


    % Exact colors from NB
    \definecolor{incolor}{rgb}{0.0, 0.0, 0.5}
    \definecolor{outcolor}{rgb}{0.545, 0.0, 0.0}



    
    % Prevent overflowing lines due to hard-to-break entities
    \sloppy 
    % Setup hyperref package
    \hypersetup{
      breaklinks=true,  % so long urls are correctly broken across lines
      colorlinks=true,
      urlcolor=urlcolor,
      linkcolor=linkcolor,
      citecolor=citecolor,
      }
    % Slightly bigger margins than the latex defaults
    
    \geometry{verbose,tmargin=1in,bmargin=1in,lmargin=1in,rmargin=1in}
    
    

    \begin{document}
    
    
    \maketitle
    
    

    
    \hypertarget{what-will-the-u.s.-geriatric-population-amount-look-like-over-time}{%
\section{What will the U.S. geriatric population amount look like over
time?}\label{what-will-the-u.s.-geriatric-population-amount-look-like-over-time}}

    The expected lifespan for the U.S. population has been increasing over
the years, due to various factors, such as improvements in medical
technology and cleaner and safer working conditions. More members of the
populace are expected to reach old age, and more are expected to be
around in the future. This leads to thinking of what impact this growing
aging population has on the rest of the populace, which in turn brings
up a more fundamental question - how will the amount of aging
individuals in the population change over time?

This model seeks to answer that question by building a population growth
model centered around age demographics. The model focuses on slices of
the population, and focuses on how the amount of individuals in each
slice changes over time. In addition to this, the probability of death
is determined for each slice of the population. The model will simulate
the growth of the population using this model for 60 years, before
determining where the state of the aging population is.

The model uses 2010 Age Demographic Census Data as the starting point
for simulation.

    \hypertarget{data-parsing}{%
\section{Data Parsing}\label{data-parsing}}

    In order to model age demographics, we follow a kind of `bucket' based
modeling, where the population is divided into buckets based on age. As
the years go by, members flow from one bucket to another. Each bucket
has an associated birth and death rate as well (the birth rate only
applies to newborn, while the death rate applies to all buckets, at
varying values) - the handling of birth and death rates is discussed in
the next section. The collection of buckets comprises the main state of
the simulation, the age demographic based population count.

Using Census data (age demographic data), we construct our initial
population bucket states. The states are organized into a Series object,
with each index corresponding to a certain age. The index is equivalent
to the age value of each bucket (the last index of the buckets is
equivalent to the cap-off age (100), where the oldest of the populace
are bundled together).

    \begin{Verbatim}[commandchars=\\\{\}]
{\color{incolor}In [{\color{incolor}1}]:} \PY{c+c1}{\PYZsh{} Configure Jupyter so figures appear in the notebook}
        \PY{o}{\PYZpc{}}\PY{k}{matplotlib} inline
\end{Verbatim}


    \begin{Verbatim}[commandchars=\\\{\}]
{\color{incolor}In [{\color{incolor}2}]:} \PY{k+kn}{import} \PY{n+nn}{pandas} \PY{k}{as} \PY{n+nn}{pd}
        \PY{k+kn}{import} \PY{n+nn}{numpy} \PY{k}{as} \PY{n+nn}{np}
        \PY{k+kn}{import} \PY{n+nn}{matplotlib}\PY{n+nn}{.}\PY{n+nn}{pyplot} \PY{k}{as} \PY{n+nn}{plt}
        \PY{k+kn}{import} \PY{n+nn}{scipy}\PY{n+nn}{.}\PY{n+nn}{io} \PY{k}{as} \PY{n+nn}{sio}
\end{Verbatim}


    \begin{Verbatim}[commandchars=\\\{\}]
{\color{incolor}In [{\color{incolor}4}]:} \PY{c+c1}{\PYZsh{} Load the initial data set, located in the same directory}
        \PY{n}{init\PYZus{}age\PYZus{}data} \PY{o}{=} \PY{n}{sio}\PY{o}{.}\PY{n}{loadmat}\PY{p}{(}\PY{l+s+s1}{\PYZsq{}}\PY{l+s+s1}{AgeDemographicData.mat}\PY{l+s+s1}{\PYZsq{}}\PY{p}{)}\PY{p}{[}\PY{l+s+s1}{\PYZsq{}}\PY{l+s+s1}{AgeDemographicData}\PY{l+s+s1}{\PYZsq{}}\PY{p}{]}
        
        \PY{c+c1}{\PYZsh{} Parse the data set \PYZhy{} use the second column for age demographic data, the first column for age value indexing}
        \PY{n}{init\PYZus{}data} \PY{o}{=} \PY{n}{pd}\PY{o}{.}\PY{n}{Series}\PY{p}{(}\PY{n}{init\PYZus{}age\PYZus{}data}\PY{p}{[}\PY{p}{:}\PY{p}{,}\PY{l+m+mi}{1}\PY{p}{]}\PY{p}{,} \PY{n}{index}\PY{o}{=}\PY{n}{init\PYZus{}age\PYZus{}data}\PY{p}{[}\PY{p}{:}\PY{p}{,}\PY{l+m+mi}{0}\PY{p}{]}\PY{p}{)}
\end{Verbatim}


    \begin{Verbatim}[commandchars=\\\{\}]
{\color{incolor}In [{\color{incolor}5}]:} \PY{n}{state} \PY{o}{=} \PY{n}{pd}\PY{o}{.}\PY{n}{Series}\PY{p}{(}\PY{p}{)}
        \PY{n}{buckets} \PY{o}{=} \PY{n}{pd}\PY{o}{.}\PY{n}{Series}\PY{p}{(}\PY{n}{index} \PY{o}{=} \PY{n}{np}\PY{o}{.}\PY{n}{linspace}\PY{p}{(}\PY{l+m+mi}{0}\PY{p}{,}\PY{l+m+mi}{100}\PY{p}{,}\PY{l+m+mi}{101}\PY{p}{,}\PY{n}{dtype}\PY{o}{=}\PY{n+nb}{int}\PY{p}{)}\PY{p}{)}
        \PY{k}{for} \PY{n}{i} \PY{o+ow}{in} \PY{n}{init\PYZus{}data}\PY{o}{.}\PY{n}{index}\PY{p}{:}
            \PY{k}{if} \PY{n}{i} \PY{o}{==} \PY{n}{init\PYZus{}data}\PY{o}{.}\PY{n}{index}\PY{p}{[}\PY{l+m+mi}{0}\PY{p}{]}\PY{p}{:}
                \PY{n}{state}\PY{p}{[}\PY{l+s+s1}{\PYZsq{}}\PY{l+s+s1}{total}\PY{l+s+s1}{\PYZsq{}}\PY{p}{]} \PY{o}{=} \PY{n}{init\PYZus{}data}\PY{p}{[}\PY{n}{i}\PY{p}{]}
                \PY{n}{k} \PY{o}{=} \PY{n}{i}
            \PY{k}{else}\PY{p}{:}
                \PY{n}{pop\PYZus{}value} \PY{o}{=} \PY{n}{init\PYZus{}data}\PY{p}{[}\PY{n}{i}\PY{p}{]}\PY{o}{/}\PY{p}{(}\PY{n}{i}\PY{o}{\PYZhy{}}\PY{n}{k}\PY{p}{)}
                \PY{k}{for} \PY{n}{j} \PY{o+ow}{in} \PY{n+nb}{range}\PY{p}{(}\PY{n}{k}\PY{p}{,} \PY{n}{i}\PY{p}{)}\PY{p}{:}
                    \PY{k}{if} \PY{n}{j} \PY{o}{\PYZlt{}}\PY{o}{=} \PY{l+m+mi}{100}\PY{p}{:}
                        \PY{n}{buckets}\PY{p}{[}\PY{n}{j}\PY{p}{]}\PY{o}{=}\PY{n}{pop\PYZus{}value}
                    \PY{k}{else}\PY{p}{:}
                        \PY{n}{buckets}\PY{p}{[}\PY{l+m+mi}{100}\PY{p}{]} \PY{o}{=} \PY{n}{pop\PYZus{}value} \PY{o}{+} \PY{n}{buckets}\PY{p}{[}\PY{l+m+mi}{100}\PY{p}{]}
                \PY{n}{k} \PY{o}{=} \PY{n}{i}
        \PY{n}{state}\PY{p}{[}\PY{l+s+s1}{\PYZsq{}}\PY{l+s+s1}{buckets}\PY{l+s+s1}{\PYZsq{}}\PY{p}{]} \PY{o}{=} \PY{n}{buckets}
\end{Verbatim}


    \hypertarget{model-construction}{%
\section{Model Construction}\label{model-construction}}

    The model's main focus is to create an update function that can handle
the change in longevity. This includes developing the concept of birth
and death rates. In this `bucket modeling' system, this means the
following: * The birth rate is assumed as dependent on a simplistic
model for birth rate - the simulation runs from the start of the data
(2010) for 60 years, so that those in the elderly range (65+) are
unaffected by the birth rate assumption. * The death rate is assumed as
constant for all ages. It will later be determined by a function of the
age group involved, and changes (increase) as the ages get older.

With this in mind, the update function has the following structure: *
Enter bucket * Check if the bucket is the last bucket: * Only apply
death rate to this bucket * Add population number from previous bucket
to this bucket * Check if the bucket is the first bucket: * Apply the
death rate to the current amount in the first bucket * Store resulting
population number into interim variable * Apply the birth rate * Set the
amount born as new value for first bucket * If neither: * Apply death
rate to the current bucket amounts * Store resulting population number
into interim variable * Receive population number of previous bucket
from another interim variable, set equal to new population bucket amount
* Exit bucket * Loop through next buckets till completion of bucket
updates

    \begin{Verbatim}[commandchars=\\\{\}]
{\color{incolor}In [{\color{incolor}204}]:} \PY{l+s+sd}{\PYZdq{}\PYZdq{}\PYZdq{}}
          \PY{l+s+sd}{`system` is an object that holds the parameters of the system:}
          \PY{l+s+sd}{    t\PYZus{}0: The start time of the simulation. This is when the data used to create the initial state was taken.}
          \PY{l+s+sd}{    t\PYZus{}end: Simulation end time.}
          \PY{l+s+sd}{    death: Array of death rates for each bucket of population. Current value was arbitrarily chosen}
          \PY{l+s+sd}{    birth: Birth rate of population. Amount born each year is pop * birth. Value is set equal to current birth rate.}
          \PY{l+s+sd}{\PYZdq{}\PYZdq{}\PYZdq{}}
          
          \PY{n}{system} \PY{o}{=} \PY{n}{pd}\PY{o}{.}\PY{n}{Series}\PY{p}{(}\PY{p}{\PYZob{}}\PY{l+s+s1}{\PYZsq{}}\PY{l+s+s1}{t\PYZus{}0}\PY{l+s+s1}{\PYZsq{}}\PY{p}{:} \PY{l+m+mi}{2010}\PY{p}{,} \PY{l+s+s1}{\PYZsq{}}\PY{l+s+s1}{t\PYZus{}end}\PY{l+s+s1}{\PYZsq{}}\PY{p}{:} \PY{l+m+mi}{2070}\PY{p}{,}
                              \PY{l+s+s1}{\PYZsq{}}\PY{l+s+s1}{death}\PY{l+s+s1}{\PYZsq{}}\PY{p}{:}\PY{n}{pd}\PY{o}{.}\PY{n}{Series}\PY{p}{(}\PY{n}{np}\PY{o}{.}\PY{n}{zeros}\PY{p}{(}\PY{l+m+mi}{101}\PY{p}{)}\PY{o}{+}\PY{l+m+mf}{0.01}\PY{p}{,} \PY{n}{index}\PY{o}{=}\PY{n}{np}\PY{o}{.}\PY{n}{linspace}\PY{p}{(}\PY{l+m+mi}{0}\PY{p}{,}\PY{l+m+mi}{100}\PY{p}{,}\PY{l+m+mi}{101}\PY{p}{,} \PY{n}{dtype}\PY{o}{=}\PY{n+nb}{int}\PY{p}{)}\PY{p}{)}\PY{p}{,}
                              \PY{l+s+s1}{\PYZsq{}}\PY{l+s+s1}{birth}\PY{l+s+s1}{\PYZsq{}}\PY{p}{:} \PY{l+m+mf}{0.012}\PY{p}{\PYZcb{}}\PY{p}{)}
          
          \PY{c+c1}{\PYZsh{}\PYZsh{}\PYZsh{}\PYZsh{} SAVE BIRTH AND DEATH RATES AS FUNCTIONS}
          \PY{c+c1}{\PYZsh{}\PYZsh{} BIRTH RATE IS BASED ON DATA}
\end{Verbatim}


    \begin{Verbatim}[commandchars=\\\{\}]
{\color{incolor}In [{\color{incolor}3}]:} \PY{k}{def} \PY{n+nf}{update\PYZus{}buckets}\PY{p}{(}\PY{n}{t}\PY{p}{,} \PY{n}{state}\PY{p}{,}\PY{n}{params}\PY{p}{)}\PY{p}{:}
            \PY{l+s+sd}{\PYZdq{}\PYZdq{}\PYZdq{}}
        \PY{l+s+sd}{    Takes current state variable containing bucket data, and updates all buckets to reflect passing of a year.}
        \PY{l+s+sd}{    }
        \PY{l+s+sd}{    Parameters:}
        \PY{l+s+sd}{        t: Time (in years) of simulation. Currently does not affect update function\PYZsq{}s behavior.}
        \PY{l+s+sd}{        state: Series object containing data about bucket states and total population}
        \PY{l+s+sd}{            buckets: Contains the actual bucket data, split into buckets per age.}
        \PY{l+s+sd}{            total: Records total population amount}
        \PY{l+s+sd}{        params: Series object containing relevant data for updating buckets.}
        \PY{l+s+sd}{            birth: Birth rate of population}
        \PY{l+s+sd}{            death: Array of death rates per bucket}
        \PY{l+s+sd}{            }
        \PY{l+s+sd}{    Returns:}
        \PY{l+s+sd}{        state: Updated state variable}
        \PY{l+s+sd}{    \PYZdq{}\PYZdq{}\PYZdq{}}
            
            \PY{c+c1}{\PYZsh{} Update the number of individuals per each bucket that died over the year}
            \PY{n}{state}\PY{p}{[}\PY{l+s+s1}{\PYZsq{}}\PY{l+s+s1}{buckets}\PY{l+s+s1}{\PYZsq{}}\PY{p}{]} \PY{o}{=} \PY{n}{state}\PY{p}{[}\PY{l+s+s1}{\PYZsq{}}\PY{l+s+s1}{buckets}\PY{l+s+s1}{\PYZsq{}}\PY{p}{]} \PY{o}{\PYZhy{}} \PY{n}{state}\PY{p}{[}\PY{l+s+s1}{\PYZsq{}}\PY{l+s+s1}{buckets}\PY{l+s+s1}{\PYZsq{}}\PY{p}{]} \PY{o}{*} \PY{n}{params}\PY{p}{[}\PY{l+s+s1}{\PYZsq{}}\PY{l+s+s1}{death}\PY{l+s+s1}{\PYZsq{}}\PY{p}{]}
            
            \PY{c+c1}{\PYZsh{} Create interim bucket state, to hold what population data needs to be updated and shifted}
            \PY{c+c1}{\PYZsh{} Shift the first 99 years of data, and add in the amount of newborns into the first slot to complete}
            \PY{n}{interim\PYZus{}buckets} \PY{o}{=} \PY{n}{pd}\PY{o}{.}\PY{n}{concat}\PY{p}{(}\PY{p}{[}\PY{n}{pd}\PY{o}{.}\PY{n}{Series}\PY{p}{(}\PY{n}{params}\PY{p}{[}\PY{l+s+s1}{\PYZsq{}}\PY{l+s+s1}{birth}\PY{l+s+s1}{\PYZsq{}}\PY{p}{]} \PY{o}{*} \PY{n}{state}\PY{p}{[}\PY{l+s+s1}{\PYZsq{}}\PY{l+s+s1}{total}\PY{l+s+s1}{\PYZsq{}}\PY{p}{]}\PY{p}{)}\PY{p}{,} 
                                        \PY{n}{state}\PY{p}{[}\PY{l+s+s1}{\PYZsq{}}\PY{l+s+s1}{buckets}\PY{l+s+s1}{\PYZsq{}}\PY{p}{]}\PY{p}{[}\PY{l+m+mi}{0}\PY{p}{:}\PY{o}{\PYZhy{}}\PY{l+m+mi}{1}\PY{p}{]}\PY{p}{]}\PY{p}{,} \PY{n}{ignore\PYZus{}index} \PY{o}{=} \PY{k+kc}{True}\PY{p}{)}
            
            \PY{c+c1}{\PYZsh{} Add the previous number of living 100+ year olds to the new number of living 100 year olds}
            \PY{n}{interim\PYZus{}buckets}\PY{p}{[}\PY{n}{interim\PYZus{}buckets}\PY{o}{.}\PY{n}{index}\PY{p}{[}\PY{o}{\PYZhy{}}\PY{l+m+mi}{1}\PY{p}{]}\PY{p}{]} \PY{o}{=} \PY{n}{interim\PYZus{}buckets}\PY{p}{[}\PY{n}{interim\PYZus{}buckets}\PY{o}{.}\PY{n}{index}\PY{p}{[}\PY{o}{\PYZhy{}}\PY{l+m+mi}{1}\PY{p}{]}\PY{p}{]} \PY{o}{+} \PY{n}{state}\PY{p}{[}\PY{l+s+s1}{\PYZsq{}}\PY{l+s+s1}{buckets}\PY{l+s+s1}{\PYZsq{}}\PY{p}{]}\PY{p}{[}\PY{o}{\PYZhy{}}\PY{l+m+mi}{1}\PY{p}{:}\PY{p}{]}
            
            \PY{c+c1}{\PYZsh{} Update the state variable}
            \PY{n}{state}\PY{p}{[}\PY{l+s+s1}{\PYZsq{}}\PY{l+s+s1}{buckets}\PY{l+s+s1}{\PYZsq{}}\PY{p}{]} \PY{o}{=} \PY{n}{interim\PYZus{}buckets}
            
            \PY{c+c1}{\PYZsh{} Calculate the new total population amount}
            \PY{n}{state}\PY{p}{[}\PY{l+s+s1}{\PYZsq{}}\PY{l+s+s1}{total}\PY{l+s+s1}{\PYZsq{}}\PY{p}{]} \PY{o}{=} \PY{n}{interim\PYZus{}buckets}\PY{o}{.}\PY{n}{sum}\PY{p}{(}\PY{p}{)}
            
            \PY{c+c1}{\PYZsh{} Return the updated state}
            \PY{k}{return} \PY{n}{state}
\end{Verbatim}


    \hypertarget{simulation}{%
\section{Simulation}\label{simulation}}

    Using the update function defined in the \textbf{Model Construction}
section, the simulation is defined as moving through a period of time,
updating the buckets with each timestep. The results of each update is
stored in a Dataframe object, so that the evolution of each individual
bucket can be seen in the end; a column is also be added for the total
population amount.

    \begin{Verbatim}[commandchars=\\\{\}]
{\color{incolor}In [{\color{incolor}215}]:} \PY{k}{def} \PY{n+nf}{run\PYZus{}simulation}\PY{p}{(}\PY{n}{state}\PY{p}{,} \PY{n}{system}\PY{p}{,} \PY{n}{update\PYZus{}func}\PY{p}{)}\PY{p}{:}
              \PY{n}{t\PYZus{}interval} \PY{o}{=} \PY{n}{system}\PY{p}{[}\PY{l+s+s1}{\PYZsq{}}\PY{l+s+s1}{t\PYZus{}end}\PY{l+s+s1}{\PYZsq{}}\PY{p}{]} \PY{o}{\PYZhy{}} \PY{n}{system}\PY{p}{[}\PY{l+s+s1}{\PYZsq{}}\PY{l+s+s1}{t\PYZus{}0}\PY{l+s+s1}{\PYZsq{}}\PY{p}{]} \PY{o}{+} \PY{l+m+mi}{1}
              \PY{n}{t\PYZus{}range} \PY{o}{=} \PY{n}{np}\PY{o}{.}\PY{n}{linspace}\PY{p}{(}\PY{n}{system}\PY{p}{[}\PY{l+s+s1}{\PYZsq{}}\PY{l+s+s1}{t\PYZus{}0}\PY{l+s+s1}{\PYZsq{}}\PY{p}{]}\PY{p}{,} \PY{n}{system}\PY{p}{[}\PY{l+s+s1}{\PYZsq{}}\PY{l+s+s1}{t\PYZus{}end}\PY{l+s+s1}{\PYZsq{}}\PY{p}{]}\PY{p}{,} \PY{n}{t\PYZus{}interval}\PY{p}{,} \PY{n}{dtype} \PY{o}{=} \PY{n+nb}{int}\PY{p}{)}
              
              \PY{n}{results} \PY{o}{=} \PY{n}{pd}\PY{o}{.}\PY{n}{DataFrame}\PY{p}{(}\PY{n}{columns} \PY{o}{=} \PY{n}{np}\PY{o}{.}\PY{n}{concatenate}\PY{p}{(}\PY{p}{(}\PY{n}{np}\PY{o}{.}\PY{n}{array}\PY{p}{(}\PY{n}{state}\PY{p}{[}\PY{l+s+s1}{\PYZsq{}}\PY{l+s+s1}{buckets}\PY{l+s+s1}{\PYZsq{}}\PY{p}{]}\PY{o}{.}\PY{n}{index}\PY{p}{)}\PY{p}{,}
                                                                \PY{n}{np}\PY{o}{.}\PY{n}{array}\PY{p}{(}\PY{l+s+s1}{\PYZsq{}}\PY{l+s+s1}{total}\PY{l+s+s1}{\PYZsq{}}\PY{p}{)}\PY{p}{)}\PY{p}{,} \PY{n}{axis} \PY{o}{=} \PY{k+kc}{None}\PY{p}{)}\PY{p}{,}
                                     \PY{n}{index} \PY{o}{=} \PY{n}{t\PYZus{}range}\PY{p}{)}
              
              \PY{k}{for} \PY{n}{t} \PY{o+ow}{in} \PY{n}{np}\PY{o}{.}\PY{n}{linspace}\PY{p}{(}\PY{n}{system}\PY{p}{[}\PY{l+s+s1}{\PYZsq{}}\PY{l+s+s1}{t\PYZus{}0}\PY{l+s+s1}{\PYZsq{}}\PY{p}{]}\PY{p}{,} \PY{n}{system}\PY{p}{[}\PY{l+s+s1}{\PYZsq{}}\PY{l+s+s1}{t\PYZus{}end}\PY{l+s+s1}{\PYZsq{}}\PY{p}{]}\PY{p}{,} \PY{n}{t\PYZus{}interval}\PY{p}{,} \PY{n}{dtype} \PY{o}{=} \PY{n+nb}{int}\PY{p}{)}\PY{p}{:}
                  \PY{n}{state} \PY{o}{=} \PY{n}{update\PYZus{}func}\PY{p}{(}\PY{n}{t}\PY{p}{,} \PY{n}{state}\PY{p}{,} \PY{n}{system}\PY{p}{)}
                  \PY{n}{results}\PY{o}{.}\PY{n}{iloc}\PY{p}{[}\PY{n}{t}\PY{o}{\PYZhy{}}\PY{n}{system}\PY{p}{[}\PY{l+s+s1}{\PYZsq{}}\PY{l+s+s1}{t\PYZus{}0}\PY{l+s+s1}{\PYZsq{}}\PY{p}{]}\PY{p}{]} \PY{o}{=} \PY{n}{pd}\PY{o}{.}\PY{n}{concat}\PY{p}{(}\PY{p}{[}\PY{n}{state}\PY{p}{[}\PY{l+s+s1}{\PYZsq{}}\PY{l+s+s1}{buckets}\PY{l+s+s1}{\PYZsq{}}\PY{p}{]}\PY{p}{,} 
                                                             \PY{n}{pd}\PY{o}{.}\PY{n}{Series}\PY{p}{(}\PY{n}{state}\PY{p}{[}\PY{l+s+s1}{\PYZsq{}}\PY{l+s+s1}{total}\PY{l+s+s1}{\PYZsq{}}\PY{p}{]}\PY{p}{)}\PY{p}{]}\PY{p}{,} 
                                                            \PY{n}{ignore\PYZus{}index} \PY{o}{=} \PY{k+kc}{True}\PY{p}{)}\PY{o}{.}\PY{n}{values}
              \PY{k}{return} \PY{n}{results}
\end{Verbatim}


    \hypertarget{results}{%
\section{Results}\label{results}}

    There are four graphs to indicate the results: the evolution of total
population over time, the evolution of the elderly population over time,
and the initial and final states of age demographics (in pie charts).
These will provide a snapshot of the model's projection for the effect
of increasing longevity on the population.

    \begin{Verbatim}[commandchars=\\\{\}]
{\color{incolor}In [{\color{incolor}259}]:} \PY{n}{res} \PY{o}{=} \PY{n}{run\PYZus{}simulation}\PY{p}{(}\PY{n}{state}\PY{p}{,} \PY{n}{system}\PY{p}{,} \PY{n}{update\PYZus{}buckets}\PY{p}{)}
          
          \PY{n}{fresh\PYZus{}pop} \PY{o}{=} \PY{n}{res}\PY{o}{.}\PY{n}{loc}\PY{p}{[}\PY{p}{:}\PY{p}{,}\PY{l+s+s1}{\PYZsq{}}\PY{l+s+s1}{0}\PY{l+s+s1}{\PYZsq{}}\PY{p}{:}\PY{l+s+s1}{\PYZsq{}}\PY{l+s+s1}{4}\PY{l+s+s1}{\PYZsq{}}\PY{p}{]}\PY{o}{.}\PY{n}{sum}\PY{p}{(}\PY{n}{axis} \PY{o}{=} \PY{l+m+mi}{1}\PY{p}{)}
          \PY{n}{student\PYZus{}pop} \PY{o}{=} \PY{n}{res}\PY{o}{.}\PY{n}{loc}\PY{p}{[}\PY{p}{:}\PY{p}{,}\PY{l+s+s1}{\PYZsq{}}\PY{l+s+s1}{5}\PY{l+s+s1}{\PYZsq{}}\PY{p}{:}\PY{l+s+s1}{\PYZsq{}}\PY{l+s+s1}{24}\PY{l+s+s1}{\PYZsq{}}\PY{p}{]}\PY{o}{.}\PY{n}{sum}\PY{p}{(}\PY{n}{axis} \PY{o}{=} \PY{l+m+mi}{1}\PY{p}{)}
          \PY{n}{young\PYZus{}pop} \PY{o}{=} \PY{n}{res}\PY{o}{.}\PY{n}{loc}\PY{p}{[}\PY{p}{:}\PY{p}{,}\PY{l+s+s1}{\PYZsq{}}\PY{l+s+s1}{25}\PY{l+s+s1}{\PYZsq{}}\PY{p}{:}\PY{l+s+s1}{\PYZsq{}}\PY{l+s+s1}{34}\PY{l+s+s1}{\PYZsq{}}\PY{p}{]}\PY{o}{.}\PY{n}{sum}\PY{p}{(}\PY{n}{axis} \PY{o}{=} \PY{l+m+mi}{1}\PY{p}{)}
          \PY{n}{middle\PYZus{}pop} \PY{o}{=} \PY{n}{res}\PY{o}{.}\PY{n}{loc}\PY{p}{[}\PY{p}{:}\PY{p}{,}\PY{l+s+s1}{\PYZsq{}}\PY{l+s+s1}{35}\PY{l+s+s1}{\PYZsq{}}\PY{p}{:}\PY{l+s+s1}{\PYZsq{}}\PY{l+s+s1}{64}\PY{l+s+s1}{\PYZsq{}}\PY{p}{]}\PY{o}{.}\PY{n}{sum}\PY{p}{(}\PY{n}{axis} \PY{o}{=} \PY{l+m+mi}{1}\PY{p}{)}
          \PY{n}{senior\PYZus{}pop} \PY{o}{=} \PY{n}{res}\PY{o}{.}\PY{n}{loc}\PY{p}{[}\PY{p}{:}\PY{p}{,}\PY{l+s+s1}{\PYZsq{}}\PY{l+s+s1}{65}\PY{l+s+s1}{\PYZsq{}}\PY{p}{:}\PY{l+s+s1}{\PYZsq{}}\PY{l+s+s1}{84}\PY{l+s+s1}{\PYZsq{}}\PY{p}{]}\PY{o}{.}\PY{n}{sum}\PY{p}{(}\PY{n}{axis} \PY{o}{=} \PY{l+m+mi}{1}\PY{p}{)}
          \PY{n}{geriatric\PYZus{}pop} \PY{o}{=} \PY{n}{res}\PY{o}{.}\PY{n}{loc}\PY{p}{[}\PY{p}{:}\PY{p}{,}\PY{l+s+s1}{\PYZsq{}}\PY{l+s+s1}{85}\PY{l+s+s1}{\PYZsq{}}\PY{p}{:}\PY{l+s+s1}{\PYZsq{}}\PY{l+s+s1}{100}\PY{l+s+s1}{\PYZsq{}}\PY{p}{]}\PY{o}{.}\PY{n}{sum}\PY{p}{(}\PY{n}{axis} \PY{o}{=} \PY{l+m+mi}{1}\PY{p}{)}
          \PY{n}{total\PYZus{}pop} \PY{o}{=} \PY{n}{res}\PY{o}{.}\PY{n}{loc}\PY{p}{[}\PY{p}{:}\PY{p}{,}\PY{l+s+s1}{\PYZsq{}}\PY{l+s+s1}{total}\PY{l+s+s1}{\PYZsq{}}\PY{p}{]}
          
          \PY{n}{plt}\PY{o}{.}\PY{n}{figure}\PY{p}{(}\PY{p}{)}
          \PY{n}{total\PYZus{}pop}\PY{o}{.}\PY{n}{plot}\PY{p}{(}\PY{p}{)}
          
          \PY{n}{plt}\PY{o}{.}\PY{n}{figure}\PY{p}{(}\PY{p}{)}
          \PY{n}{geriatric\PYZus{}pop}\PY{o}{.}\PY{n}{plot}\PY{p}{(}\PY{p}{)}
          
          \PY{n}{plt}\PY{o}{.}\PY{n}{figure}\PY{p}{(}\PY{p}{)}
          \PY{n}{plt}\PY{o}{.}\PY{n}{pie}\PY{p}{(}\PY{p}{[}\PY{n}{fresh\PYZus{}pop}\PY{p}{[}\PY{n}{fresh\PYZus{}pop}\PY{o}{.}\PY{n}{index}\PY{p}{[}\PY{l+m+mi}{0}\PY{p}{]}\PY{p}{]}\PY{p}{,} \PY{n}{student\PYZus{}pop}\PY{p}{[}\PY{n}{student\PYZus{}pop}\PY{o}{.}\PY{n}{index}\PY{p}{[}\PY{l+m+mi}{0}\PY{p}{]}\PY{p}{]}\PY{p}{,}
                   \PY{n}{young\PYZus{}pop}\PY{p}{[}\PY{n}{young\PYZus{}pop}\PY{o}{.}\PY{n}{index}\PY{p}{[}\PY{l+m+mi}{0}\PY{p}{]}\PY{p}{]}\PY{p}{,} \PY{n}{middle\PYZus{}pop}\PY{p}{[}\PY{n}{middle\PYZus{}pop}\PY{o}{.}\PY{n}{index}\PY{p}{[}\PY{l+m+mi}{0}\PY{p}{]}\PY{p}{]}\PY{p}{,}
                   \PY{n}{senior\PYZus{}pop}\PY{p}{[}\PY{n}{senior\PYZus{}pop}\PY{o}{.}\PY{n}{index}\PY{p}{[}\PY{l+m+mi}{0}\PY{p}{]}\PY{p}{]}\PY{p}{,} \PY{n}{geriatric\PYZus{}pop}\PY{p}{[}\PY{n}{geriatric\PYZus{}pop}\PY{o}{.}\PY{n}{index}\PY{p}{[}\PY{l+m+mi}{0}\PY{p}{]}\PY{p}{]}\PY{p}{]}\PY{p}{,}
                  \PY{n}{labels} \PY{o}{=} \PY{p}{[}\PY{l+s+s1}{\PYZsq{}}\PY{l+s+s1}{fresh}\PY{l+s+s1}{\PYZsq{}}\PY{p}{,} \PY{l+s+s1}{\PYZsq{}}\PY{l+s+s1}{students}\PY{l+s+s1}{\PYZsq{}}\PY{p}{,} \PY{l+s+s1}{\PYZsq{}}\PY{l+s+s1}{young}\PY{l+s+s1}{\PYZsq{}}\PY{p}{,} \PY{l+s+s1}{\PYZsq{}}\PY{l+s+s1}{middle}\PY{l+s+s1}{\PYZsq{}}\PY{p}{,} \PY{l+s+s1}{\PYZsq{}}\PY{l+s+s1}{senior}\PY{l+s+s1}{\PYZsq{}}\PY{p}{,} \PY{l+s+s1}{\PYZsq{}}\PY{l+s+s1}{old}\PY{l+s+s1}{\PYZsq{}}\PY{p}{]}\PY{p}{)}
          \PY{n}{plt}\PY{o}{.}\PY{n}{title}\PY{p}{(}\PY{l+s+s1}{\PYZsq{}}\PY{l+s+s1}{Population demographic at start of simulation}\PY{l+s+s1}{\PYZsq{}}\PY{p}{)}
          
          \PY{n}{plt}\PY{o}{.}\PY{n}{figure}\PY{p}{(}\PY{p}{)}
          \PY{n}{plt}\PY{o}{.}\PY{n}{pie}\PY{p}{(}\PY{p}{[}\PY{n}{fresh\PYZus{}pop}\PY{p}{[}\PY{n}{fresh\PYZus{}pop}\PY{o}{.}\PY{n}{index}\PY{p}{[}\PY{o}{\PYZhy{}}\PY{l+m+mi}{1}\PY{p}{]}\PY{p}{]}\PY{p}{,} \PY{n}{student\PYZus{}pop}\PY{p}{[}\PY{n}{student\PYZus{}pop}\PY{o}{.}\PY{n}{index}\PY{p}{[}\PY{o}{\PYZhy{}}\PY{l+m+mi}{1}\PY{p}{]}\PY{p}{]}\PY{p}{,}
                   \PY{n}{young\PYZus{}pop}\PY{p}{[}\PY{n}{young\PYZus{}pop}\PY{o}{.}\PY{n}{index}\PY{p}{[}\PY{o}{\PYZhy{}}\PY{l+m+mi}{1}\PY{p}{]}\PY{p}{]}\PY{p}{,} \PY{n}{middle\PYZus{}pop}\PY{p}{[}\PY{n}{middle\PYZus{}pop}\PY{o}{.}\PY{n}{index}\PY{p}{[}\PY{o}{\PYZhy{}}\PY{l+m+mi}{1}\PY{p}{]}\PY{p}{]}\PY{p}{,}
                   \PY{n}{senior\PYZus{}pop}\PY{p}{[}\PY{n}{senior\PYZus{}pop}\PY{o}{.}\PY{n}{index}\PY{p}{[}\PY{o}{\PYZhy{}}\PY{l+m+mi}{1}\PY{p}{]}\PY{p}{]}\PY{p}{,} \PY{n}{geriatric\PYZus{}pop}\PY{p}{[}\PY{n}{geriatric\PYZus{}pop}\PY{o}{.}\PY{n}{index}\PY{p}{[}\PY{o}{\PYZhy{}}\PY{l+m+mi}{1}\PY{p}{]}\PY{p}{]}\PY{p}{]}\PY{p}{,}
                  \PY{n}{labels} \PY{o}{=} \PY{p}{[}\PY{l+s+s1}{\PYZsq{}}\PY{l+s+s1}{fresh}\PY{l+s+s1}{\PYZsq{}}\PY{p}{,} \PY{l+s+s1}{\PYZsq{}}\PY{l+s+s1}{students}\PY{l+s+s1}{\PYZsq{}}\PY{p}{,} \PY{l+s+s1}{\PYZsq{}}\PY{l+s+s1}{young}\PY{l+s+s1}{\PYZsq{}}\PY{p}{,} \PY{l+s+s1}{\PYZsq{}}\PY{l+s+s1}{middle}\PY{l+s+s1}{\PYZsq{}}\PY{p}{,} \PY{l+s+s1}{\PYZsq{}}\PY{l+s+s1}{senior}\PY{l+s+s1}{\PYZsq{}}\PY{p}{,} \PY{l+s+s1}{\PYZsq{}}\PY{l+s+s1}{old}\PY{l+s+s1}{\PYZsq{}}\PY{p}{]}\PY{p}{)}
          \PY{n}{plt}\PY{o}{.}\PY{n}{title}\PY{p}{(}\PY{l+s+s1}{\PYZsq{}}\PY{l+s+s1}{Population demographics after 60 years}\PY{l+s+s1}{\PYZsq{}}\PY{p}{)}
\end{Verbatim}


\begin{Verbatim}[commandchars=\\\{\}]
{\color{outcolor}Out[{\color{outcolor}259}]:} Text(0.5,1,'Population demographics after 60 years')
\end{Verbatim}
            
    \begin{center}
    \adjustimage{max size={0.9\linewidth}{0.9\paperheight}}{output_17_1.png}
    \end{center}
    { \hspace*{\fill} \\}
    
    \begin{center}
    \adjustimage{max size={0.9\linewidth}{0.9\paperheight}}{output_17_2.png}
    \end{center}
    { \hspace*{\fill} \\}
    
    \begin{center}
    \adjustimage{max size={0.9\linewidth}{0.9\paperheight}}{output_17_3.png}
    \end{center}
    { \hspace*{\fill} \\}
    
    \begin{center}
    \adjustimage{max size={0.9\linewidth}{0.9\paperheight}}{output_17_4.png}
    \end{center}
    { \hspace*{\fill} \\}
    
    \hypertarget{validation}{%
\section{Validation}\label{validation}}

    Because this model deals in the space of future population projection,
there are two ways to validate the model: start simulation in the past,
such that it can be compared with actual data, or compare the data
against projections provided by institutes such as the US Census or the
UN DESA. These will provide a sense of the model's reasonability.

    \hypertarget{interpretation}{%
\section{Interpretation}\label{interpretation}}

    The model's results are indicative of the effects of increasing
longevity on the population. Compared to Japan, which is already
experiencing the effects of an `aging population', the model shows that
----. In addition to simply predicting the number of elderly that will
be in the world, the model also serves a basis for
improvements/alterations in public policy geared towards and effecting
the elderly population, such as policies for Social Security funds
management, and hospital grant allocation.


    % Add a bibliography block to the postdoc
    
    
    
    \end{document}
